\documentclass[a4paper]{scrartcl}
\usepackage[ngerman]{babel}
\usepackage[utf8]{inputenc}
\usepackage[T1]{fontenc}
\usepackage{lmodern}
\usepackage{amssymb}
\usepackage{amsmath}
\usepackage{enumerate}
\usepackage{pgfplots}
\usepackage{scrpage2}\pagestyle{scrheadings}
\usepackage{tikz}
\usetikzlibrary{patterns}

\newcommand{\titleinfo}{Hausaufgaben zum 24. Mai 2012}

\title{\titleinfo}
\author{Elena Noll, Sven-Hendrik Haase, Arne Struck}
\date{\today}
\ihead{EN, SHH, AS}
\chead{\titleinfo}
\ohead{\today}
\setheadsepline{1pt}
\newcommand{\qed}{\quad \square}

\begin{document}
\maketitle

\begin{enumerate}

\item[\textbf{1.}]
\begin{enumerate}
\item
Es gilt: \(x_i = \frac {i(b-a)} n\) sowie \(\sum\limits_{i=1}^\infty i^3 = \frac {n^2(n+1)^2} 4\).
\begin{align}
O_n =& \frac {b-a} n \cdot \sum\limits_{i=1}^n f(x_i) \\
    =& \frac {b-a} n \cdot \sum\limits_{i=1}^n f\Bigg(\frac {i(b-a)} n\Bigg) \\
    =& \frac 3 n \sum\limits_{i=1}^n f\Bigg(\frac {3i} n \Bigg)^3 \\
    =& \frac {27} {n^3} \cdot \frac 3 n \sum\limits_{i=1}^n i^3 \\
    =& \frac {81} {n^4} \cdot \frac {n^2(n+1)^2} 4 \\
    =& \frac {81n^4+2n+1} {4n^4} \\
    =& \lim\limits_{n\to\infty} \Bigg(\frac {81n^4+2n+1} {4n^4}\Bigg) \\
    =& \lim\limits_{n\to\infty} \Bigg(\frac {n^4(81+\frac {2n} {n^4}+\frac 1 {n^4})} {n^4 (4)}\Bigg) \\
    =& \frac {81} 4
\end{align}

\item
\begin{align}
F(x) &= \frac 1 4 x^4 \\
\int_0^3 f(x)dx &= F(b) - F(a) \\
&= \Bigg(\frac 1 4 \cdot 3^4\Bigg) - \Bigg(\frac 1 4 \cdot 0^4\Bigg) \\
&= \frac {81} 4
\end{align}
\end{enumerate}

\item[\textbf{2.}]
\begin{enumerate}[(i)]

\item
~\\
\begin{align}
f(x) &= x^2-x-6 \\
F(x) &= \frac 1 3 x^3 - \frac 1 2 x^2 - 6x \\
\int_1^3 f(x) dx &= F(3) - F(1) \\
&= (\frac 1 3 \cdot 3^3 - \frac 1 2 \cdot 3^2 - 6 \cdot 2) \\
&= (9 - \frac 9 2 - 18) - (- \frac {37} 6) \\
&= - \frac {22} 3
\end{align}
\centerline{
\begin{tikzpicture}
  \begin{axis}[xmin=0, xmax=4, 
    xlabel=$x$,
    ylabel={$f(x) = x^2 - x - 6$}
  ] 
    \addplot[fill=blue,fill opacity=0.3,domain=1:3]{x^2 - x - 6} \closedcycle;
    \addplot[smooth]{x^2 - x - 6};
  \end{axis}
\end{tikzpicture}
}

\newpage
\item
~\\
\begin{align}
f(x) &= \sqrt[3]{x} = x^{\frac 1 3} \\
F(x) &= \frac 3 4 x^{\frac 4 3} \\
\int_1^3 f(x) dx &= F(3) - F(1) \\
&= (\frac 3 4 x^{\frac 4 3}) - (\frac 3 4 \cdot 1^{\frac 4 3}) \\
&\approx 2.495
\end{align}
\centerline{
\begin{tikzpicture}
  \begin{axis}[xmin=0, xmax=4, 
    xlabel=$x$,
    ylabel={$f(x) = \sqrt[3]{x}$}
  ] 
    \addplot[fill=blue,fill opacity=0.3,domain=1:3]{x^(1/3)} \closedcycle;
    \addplot[smooth]{x^(1/3)};
  \end{axis}
\end{tikzpicture}
}

\newpage
\item
~\\
\begin{align}
f(x) &= \frac 1 {1+x^2} = 1 \cdot (1+x^2)^{-1} \\
F(x) &= \tan^{-1}(x) \\
\int_1^3 f(x) dx &= F(3) - F(1) \\
&= \tan^{-1}(3) - tan^{-1}(1) \\
&\approx 26.565
\end{align}
\centerline{
\begin{tikzpicture}
  \begin{axis}[xmin=0, xmax=4, 
    xlabel=$x$,
    ylabel={$f(x) = \frac 1 {1+x^2}$}
  ] 
    \addplot[fill=blue,fill opacity=0.3,domain=1:3]{1/(1+x^2)} \closedcycle;
    \addplot[smooth]{1/(1+x^2)};
  \end{axis}
\end{tikzpicture}
}

\newpage
\item
~\\
\begin{align}
f(x) &= \ln x \\
F(x) &= x\ln x - \int 1 dx \\
&= x \ln x - x \\
&= x(\ln x - 1) \\
\int_1^3 f(x)dx &= F(3) - F(1) \\
&= 3(\ln 3 -1) - 1(\ln 1 - 1) \\
&\approx 1.2958
\end{align}
\centerline{
\begin{tikzpicture}
  \begin{axis}[xmin=0, xmax=4, 
    xlabel=$x$,
    ylabel={$f(x) = ln(x)$}
  ] 
    \addplot[fill=blue,fill opacity=0.3,domain=1:3]{ln x} \closedcycle;
    \addplot[smooth]{ln x};
  \end{axis}
\end{tikzpicture}
}

\newpage
\item
~\\
\begin{align}
f(x) &= e^{-x} \\
F(x) &= -e \\
\int_1^3 f(x) dx &= F(3) - F(1) \\
&= -e^{-3} - (-e^{-1}) \\
&\approx 0.3181
\end{align}
\centerline{
\begin{tikzpicture}
  \begin{axis}[xmin=0, xmax=4, 
    xlabel=$x$,
    ylabel={$f(x) = e^{-x}$}
  ] 
    \addplot[fill=blue,fill opacity=0.3,domain=1:3]{e^(-x)} \closedcycle;
    \addplot[domain=0:4]{e^(-x)};
  \end{axis}
\end{tikzpicture}
}

\end{enumerate}

\item[\textbf{3.}]
\begin{enumerate}[(i)]
\item
\begin{align}
\int(x^4+2x^3-x+5) \ dx 
    &= \frac{1}{5}x^5+\frac{1}{2}x^4-\frac{1}{2}x^2+5x
\end{align}
\item
\begin{align}
\int\frac{1}{\sqrt{x^3}} \ dx 
	&= \int x^{-\frac{3}{2}} \ dx \ (für x>0)\\
	&=-\frac{2}{3}x^{-\frac{5}{2}}
\end{align}
\item
\begin{align}
\int x\cdot\sin(3x) \ dx 
	&= -\frac{1}{3} \cos(3x)\cdot x -\int -\frac{1}{3}\cos (3x) \ dx\\
	&= -\frac{x}{3} \cos(3x)- \Big(-\frac{1}{9} \sin(3x) \Big)\\
	&= -\frac{x}{3} \cos(3x)+ \frac{1}{9}\sin(3x)
\end{align}
Probe:
\begin{align}
\Big(-\frac{x}{3} \cos(3x)+ \frac{1}{9}\sin(3x)\Big)'
	&=\Big(-\frac{x}{3} \cos(3x)\Big)'+\Big(\frac{1}{9}\sin(3x)\Big)'\\
	&=\Big(-\frac{x}{3} \cos(3x)\Big)'+\frac{1}{3}\cos(3x)\\
	&=\Big(-\frac{1}{3}\cos(3x)\Big)+(x\cdot \sin(3x))+\frac{1}{3}\cos(3x)\\
	&=x\cdot \sin(3x)
\end{align}
\newpage
\item
\begin{align}
\int x^3\ln x\ dx
	&=\frac{1}{4}x^4 \ln x-\int \frac{1}{x} \cdot\frac{1}{4} x^4 \ dx\\
	&=\frac{1}{4}x^4 \ln x-\int \frac{x^3}{4}\ dx\\
	&=\frac{1}{4}x^4 \ln x-\frac{1}{16} x^4\\
	&=\frac{x^4}{4}\Big(\ln x-\frac{1}{4}\Big)\\
\end{align}
Probe:
\begin{align}
\Big(\frac{x^4}{4}\Big(\ln x-\frac{1}{4}\Big)\Big)'
	&=\Big(\frac{x^4}{4}\ln x -\frac{x^4}{16}\Big)'\\
	&=\Big(\frac{x^4}{4}\ln x\Big)'-\Big(\frac{x^4}{16}\Big)'\\
	&=x^3 \ln x+ \frac{1}{x}\cdot \frac{x^4}{4}-\frac{x^3}{4}\\
	&=x^3 \ln x
\end{align}
\item
\begin{align}
\int x^2 e^x\ dx
	&=e^x x^2 - \int 2x e^x\ dx\\
\int e^x 2x\ dx
	&=e^x 2x - \int 2e^x\ dx\\
	&=e^x 2x- 2e^x\\
\int x^2 e^x\ dx
	&=e^x x^2 - (e^x 2x- 2e^x) \\
	&=e^x (x^2 -2x+2) 
\end{align}
Probe:
\begin{align}
(e^x(x^2-2x+2))'
	&=(x^2 e^x-2x e^x+2 e^x)'\\
	&=(x^2 e^x)'-(2x e^x)'+(2e^x)'\\
	&=x^2 e^x +2x e^x -2e^x -2x e^x+2e^x\\
	&=x^2 e^x
\end{align}
\end{enumerate}
\newpage

\item[\textbf{4.}]

\item[\textbf{5.}]
\(f:[0,3]\ \rightarrow \mathbb{R}\)\\
\(f(x)=7x^3-42x^2+63x-2\)\\
\(f'(x)=21x^2-84x+63\)\\
\(f''(x)=42x-84\)\\
\(F(x) = \int f(x)\ dx= \frac{7}{4}x^4 -14x^3 +\frac{63}{2}x^2-2x\)\\
\begin{enumerate}[(i)]
\item
Die Tageshöchsttemperatur ist das Größte Maximum im Intervall [0,3].\\
\begin{align}
f'(x)=0&=21x^2-84x+63\\
	&=x^2-4x+3
\end{align}
Lösung per pq-Formel:
\begin{align}
x_{1/2} &=-\Big(-\frac{4}{2}\Big)\pm \sqrt{\Big(\frac{4}{2}\Big)^2 -3} \\
	&=2 \pm 1\\
	\Rightarrow x_1&=1 \vee x_2=3
\end{align}
\begin{align}
f''(1) &= 42-84 = -42<0 \Rightarrow Maximum\\
f''(3) &= 42\cdot 3 - 84 =42>0 \Rightarrow Minimum\\
\end{align}
\(\Rightarrow\) Die Höchsttemperatur liegt bei x=1.
\begin{center}
\(f(1)=7-42+63-2=26\)
\end{center}
\(\Rightarrow\) Die Höchsttemperatur ist 26\(^\circ \) Celsius.\\
\item
Ähnlich, wie bei (i) liegt die Tagestiefsttemperatur beim Minimum im Intervall [0,3].\\
Aus (i) ist ersichtlich, dass das Minimum bei x=3 liegt.
\begin{align}
f(3) &= 7\cdot 3^3-42\cdot 3^2 + 63 \cdot 3 -2\\
	&= 3(7\cdot 9 - 42\cdot 3 +63) -2\\
	&= 3 \cdot 0 -2\\
	&= -2
\end{align}
\(\Rightarrow\) Die Tiefsttemperatur ist -2\(^\circ \)Celsius.\\
\item
Die Durchschnittstemperatur des Tages ist gleichzusetzen mit dem Mittelwert \(\overline{f(x)}\) im Intervall [0,3].\\
\begin{align}
\overline{f(x)}&= \frac{1}{3-0} \cdot \int_0^3 f(x)\ dx\\
	&= \frac{1}{3-0} \cdot (F(3)-F(0))\\
	&= \frac{1}{3-0} \cdot \bigg( \Big( \frac{7}{4}\cdot 3^4 -14\cdot 3^3 +\frac{63}{2} \cdot 3^2 -2\cdot 3 \Big) -0 \bigg)\\
	&= \frac{1}{3-0} \cdot 41,25\\
	&= 13,75	
\end{align}
\(\Rightarrow \) Die Durchschnittstemperatur ist 13,75\(^\circ\) Celsius.
\end{enumerate}

\end{enumerate}
%Ende aller Aufgaben
\end{document}

