\documentclass[a4paper]{scrartcl}
\usepackage[ngerman]{babel}
\usepackage[utf8]{inputenc}
\usepackage[T1]{fontenc}
\usepackage{lmodern}
\usepackage{amssymb}
\usepackage{amsmath}
\usepackage{enumerate}
\usepackage{scrpage2}\pagestyle{scrheadings}
\usepackage{tikz}

\newcommand{\titleinfo}{Hausaufgaben zum 12./13. Januar 2012}

\title{\titleinfo}
\author{Elena Noll, Sven-Hendrik Haase, Arne Struck}
\date{\today}
\ihead{EN, SHH, AF}
\chead{\titleinfo}
\ohead{\today}
\setheadsepline{1pt}
\newcommand{\qed}{\quad \square}

\begin{document}
\maketitle

\begin{enumerate}
\item[\textbf{1.}]
\begin{enumerate}[1)]
\item

\[1. Fall: x > -5 \Rightarrow Nenner positiv\]
\begin{align}
\frac 2 {x+5} \geq 3 & \Leftrightarrow 2 \leq 3x+15\\
                   & \Leftrightarrow -13 \leq 3x\\
                   & \Leftrightarrow -\frac {13} 3 \leq x
\end{align} 

\[2. Fall: x < -5 \Rightarrow Nenner negativ\]
\begin{align}
\frac 2 {x-5} \geq 3 & \Leftrightarrow 2 \geq 3x+15\\
                   & \Leftrightarrow -13 \geq 3x\\
                   & \Leftrightarrow -\frac {13} 3 \geq x
\end{align}

\[L = (-5, -\frac {13} 3]\]

\newpage
\item
\[|3x-4| \geq 2 : x \in \mathbb{R} \] 
\[1. Fall: 3 > \frac 4 3\]
\begin{align}
|3x-4| \geq 2 & \Leftrightarrow 3x-4 \geq 2\\
              & \Leftrightarrow 3x \geq 6\\
              & \Leftrightarrow x \geq 2
\end{align}

\[2. Fall: x < \frac 4 3 \]
\begin{align}
|3x-4| \geq 2 & \Leftrightarrow -(3x-4) \geq 2\\
              & \Leftrightarrow -3x+4 \geq 2\\
              & \Leftrightarrow -3x \geq -2\\
              & \Leftrightarrow x \leq \frac 2 3
\end{align}

\[L = (-\infty, \frac 2 3] \cup [2, +\infty) \]


\item
\begin{enumerate}[a)]
\item
\begin{align}
|a_n - a| \Leftrightarrow | \frac {3n+2} {n+4} -3 | & = | \frac {3n+2-3n-12} {n+4} | \\
&= | -\frac {10} {n+4} | = \frac {10} {n+4}
\end{align}

\item
\[ | a_n - a | < \epsilon \]
\begin{align}
&\Leftrightarrow \frac {10} {n+4} < \epsilon\\
&\Leftrightarrow \frac {10} \epsilon < n+4\\
&\Leftrightarrow \frac {10} \epsilon -4 < n\\
&\Rightarrow N > \frac {10} \epsilon -4
\end{align}

\item
\begin{align}
&\epsilon = \frac 1 5 : N > 10*5-4 = 46 \Rightarrow N = 47\\
&\epsilon = \frac 1 {100} : N > 1000-4 = 996 \Rightarrow N = 997\\
&\epsilon = \frac 1 {1000} : N > 10000-4 = 9996 \Rightarrow N = 9997 
\end{align}
\end{enumerate}

\newpage
\item
Nachweis der Beschränktheit.

Induktionsanfang:
\[1 \leq 1 < 2\]

Induktionsannahme:
Für ein beliebiges aber fest gewähltes \( n \in \mathbb{N}\) gilt:
\[ a_{n+1} = (\frac {a_n} 2)^2 + 1\]
\[ 1 \leq a_{n+1} < 2 \]

Induktionsschritt:
\[a_{(n+1)+1} =  ( \frac {a_{n+1}} 2 )^2 + 1 =^{IA} (\frac {(\frac {a_n} 2)^2 + 1} 2)^2 + 1 \]

Daraus folgt:

Nach Induktionsannahme ist der Zähler < 2. Hierraus folgt: Bruch < 1.
Aus alledem folgt, dass die Gleichung < 2 sein muss.


\end{enumerate}
%Ende aller Aufgaben
\end{enumerate}
\end{document}

