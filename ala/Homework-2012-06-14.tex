	\documentclass[a4paper]{scrartcl}
\usepackage[ngerman]{babel}
\usepackage[utf8]{inputenc}
\usepackage[T1]{fontenc}
\usepackage{lmodern}
\usepackage{amssymb}
\usepackage{amsmath}
\usepackage{enumerate}
\usepackage{pgfplots}
\usepackage{scrpage2}\pagestyle{scrheadings}
\usepackage{tikz}
\usetikzlibrary{patterns}

\newcommand{\titleinfo}{Hausaufgaben zum 7. Juni 2012}

\title{\titleinfo}
\author{Elena Noll, Sven-Hendrik Haase, Arne Struck}
\date{\today}
\ihead{EN, SHH, AS}
\chead{\titleinfo}
\ohead{\today}
\setheadsepline{1pt}
\setcounter{secnumdepth}{0}
\newcommand{\qed}{\quad \square}

\begin{document}
\maketitle
\notag

\section{1.}
\subsection{a)}
\subsection{b)}
\subsection{c)}

\section{2.}
\subsection{(i)}
Nach Quotientenkriterium:
\[ \lim_{i\to\infty} \bigg|\frac {-1} {2^{i+2}} \cdot \frac {2^{i+1}} {-1}\bigg| = \frac 1 2 < 1 \]
Also konvergiert die Summe. \\
Grenzwert: 
\[ \sum_{i=1}^\infty \frac {-1} {2^{i+1}} = - \Big(\frac 1 {2^i} + \frac 1 2 \Big) = - \frac 1 2 \]

\subsection{(ii)}
Nach Leibniz-Kriterium:
\[ (-1)^i \cdot \frac i {2(i+1)} \]
mit \(a_i = \frac i {2(i+1)} \).
\[ \lim_{i\to\infty} a_i = \frac 1 2 \neq 0 \]
Also divergiert die Summe.

\subsection{(iii)}
Es handelt sich um eine harmonische Reihe, die aber kleiner als die harmonische Reihe \( \frac 1 n \) ist. Daraus kann man per Majorantenkriterium folgern, dass diese Summe auch divergieren muss.

\subsection{(iv)}
Nach Quotientenkriterium:
\[ \lim_{i\to\infty} \bigg| \frac {(-1)^{i+2}} {2^{i+1}} \cdot \frac {2^i} {(-1)^{i+1}} \bigg| = \frac 1 2 \le 1 \]
Also konvergiert die Summe. \\
Grenzwert:
\[ \sum_{i=1}^\infty \frac {{-1}^{i+1}} {2^i} = \sum_{i=1}^\infty -\Big(-\frac 1 2\Big)^i \]
\[ 1+\sum_{i=0}^\infty -\Big(-\frac 1 2\Big)^i = 1 -\frac 1 {1-(-\frac 1 2)} = \frac 1 3 \]

\subsection{(v)}
Nach Leibniz-Kriterium:
\[ \lim_{i\to\infty} \frac 1 {2i+1} \rightarrow 0 \]
Also konvergiert die Summe.

\subsection{(vi)}
Nach Leibniz-Kriterium:
\[ \lim_{i\to\infty} \frac 1 {2i} - 1 \rightarrow 0 \]
Also konvergiert die Summe.

\section{3.}

\section{4.}
\subsection{a)}
\subsection{b)}

\end{document}

