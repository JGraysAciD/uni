	\documentclass[a4paper]{scrartcl}
\usepackage[ngerman]{babel}
\usepackage[utf8]{inputenc}
\usepackage[T1]{fontenc}
\usepackage{lmodern}
\usepackage{amssymb}
\usepackage{amsmath}
\usepackage{enumerate}
\usepackage{pgfplots}
\usepackage{scrpage2}\pagestyle{scrheadings}
\usepackage{tikz}
\usetikzlibrary{patterns}

\newcommand{\titleinfo}{Hausaufgaben zum 7. Juni 2012}

\title{\titleinfo}
\author{Elena Noll, Sven-Hendrik Haase, Arne Struck}
\date{\today}
\ihead{EN, SHH, AS}
\chead{\titleinfo}
\ohead{\today}
\setheadsepline{1pt}
\setcounter{secnumdepth}{0}
\newcommand{\qed}{\quad \square}

\begin{document}
\maketitle
\notag

\section{1.}
\subsection{a)}
Nach Wurzelkriterium:
\[ \lim_{i\to\infty} \sqrt[i]{\bigg| \frac{i}{2^i} \bigg|} = 1/2 * \sqrt[i]{i} = \frac{1} {2} < 1 \]
Also konvergiert die Summe.

\subsection{b)}
Nach Quotientenkriterium:
\begin{align}
&\lim_{i\to\infty} \bigg| \frac{(-1)^{(i+1)} * (i+1)! * i^i} {(i+1)^{(i+1)} * (-1)^i * i!} \bigg| \\
&\lim_{i\to\infty} \bigg| - \frac{(i+1)!} {(i+1)^{(i+1)}} * \frac{i^i} {i!} \bigg|
\end{align}
Da \(\lim_{i\to\infty} \frac{(i+1)!} {(i+1)^{(i+1)}} = 0\) gilt, ist auch der Limes der ganzen Gleichung 0. \\
Somit ist der Limes < 1 und es ist gezeigt, dass Konvergenz vorliegt. 

\subsection{c)}
\subsubsection{(i)}
\begin{align}
  &\lim_{i\to\infty} \bigg| \frac{(i+1)^2 * 2^{(i+1)} * x^{(i+1)}} {i^2 * 2^i * x^i} \bigg| \\
= &\lim_{i\to\infty} \bigg| \frac{(i+1)^2} {i^2} * 2x \bigg| \\
= &\lim_{i\to\infty} \bigg| \frac{(i+1)^2} {i^2} \bigg| * |2x| \\
= & 2|x| * \lim_{i\to\infty} \bigg| \frac{i^2 + 2i + 2} {i^2} \bigg| \\
= & 2|x|
\end{align}
Daraus folgt: 
\[2|x| < 1 \Leftrightarrow |x| < \frac{1}{2}\]
\[2|x| > 1 \Leftrightarrow |x| >s \frac{1}{2}\]
Daraus folgt: \\
\[R = \frac{1}{2}\]

\subsubsection{(ii)}
\[lim_{i\to\infty} \sqrt[i]{|i^2 * 2^i * x^i} = 2|x| * lim_{i\to\infty} \sqrt[i]{i^2} = 2|x|\]
Daraus folgt: 
\[2|x| < 1 \Leftrightarrow |x| < \frac{1}{2}\]
\[2|x| > 1 \Leftrightarrow |x| >s \frac{1}{2}\]
Daraus folgt: \\
\[R = \frac{1}{2}\]
\section{2.}
\subsection{(i)}
Nach Quotientenkriterium:
\[ \lim_{i\to\infty} \bigg|\frac {-1} {2^{i+2}} \cdot \frac {2^{i+1}} {-1}\bigg| = \frac 1 2 < 1 \]
Also konvergiert die Summe. \\
Grenzwert: 
\[ \sum_{i=1}^\infty \frac {-1} {2^{i+1}} = - \Big(\frac 1 {2^i} + \frac 1 2 \Big) = - \frac 1 2 \]

\subsection{(ii)}
Nach Leibniz-Kriterium:
\[ (-1)^i \cdot \frac i {2(i+1)} \]
mit \(a_i = \frac i {2(i+1)} \).
\[ \lim_{i\to\infty} a_i = \frac 1 2 \neq 0 \]
Also divergiert die Summe.

\subsection{(iii)}
Es handelt sich um eine harmonische Reihe, die aber kleiner als die harmonische Reihe \( \frac 1 n \) ist. Daraus kann man per Majorantenkriterium folgern, dass diese Summe auch divergieren muss.

\subsection{(iv)}
Nach Quotientenkriterium:
\[ \lim_{i\to\infty} \bigg| \frac {(-1)^{i+2}} {2^{i+1}} \cdot \frac {2^i} {(-1)^{i+1}} \bigg| = \frac 1 2 \le 1 \]
Also konvergiert die Summe. \\
Grenzwert:
\[ \sum_{i=1}^\infty \frac {{-1}^{i+1}} {2^i} = \sum_{i=1}^\infty -\Big(-\frac 1 2\Big)^i \]
\[ 1+\sum_{i=0}^\infty -\Big(-\frac 1 2\Big)^i = 1 -\frac 1 {1-(-\frac 1 2)} = \frac 1 3 \]

\subsection{(v)}
Nach Leibniz-Kriterium:
\[ \lim_{i\to\infty} \frac 1 {2i+1} \rightarrow 0 \]
Also konvergiert die Summe.

\subsection{(vi)}
Nach Leibniz-Kriterium:
\[ \lim_{i\to\infty} \frac 1 {2i} - 1 \rightarrow 0 \]
Also konvergiert die Summe.

\section{3.}

\section{4.}
\subsection{a)}
Dies lässt sich per asymptotischer Analyse zeigen. \footnote{http://www.mathi.uni-heidelberg.de/\textasciitilde thaeter/surprises/euler.pdf}
Demnach gilt:

\[H_n=\sum_{k=1}^n \frac{1}{k} = \gamma + \ln (n) +\mathcal O\!\left(\frac 1{n}\right),\quad n\to\infty \]
%
\(\gamma\) ist dabei wie folgt definitiert (Euler-Mascheroni-Konstante):

\[ \gamma = \lim_{n\to\infty} \left(H_n - \ln n\right) = \lim_{n \rightarrow \infty } \left(\sum_{k=1}^n \frac{1}{k} - \ln(n) \right)=\int_1^\infty\left({1\over\lfloor x\rfloor}-{1\over x}\right)\,\mathrm dx \] \\
%
Bei \(\mathcal O\!\left(\frac 1{n}\right)\) handelt es sich um den Restterm, der extrem klein ist und für die Aufgaben nicht allzu relevant ist. \\ \\
%
Umgeschrieben ergibt sich also für \(H_n -1 \leq \ln (n) \leq H_n \):
\[ \gamma + \ln (n) +\mathcal O\!\left(\frac 1{n}\right) -1 \leq \ln (n) \leq \gamma + \ln (n) +\mathcal O\!\left(\frac 1{n}\right)\]
%
Hieran lässt sich nun leicht sehen, dass \( \ln (n) \leq \gamma + \ln (n) +\mathcal O\!\left(\frac 1{n}\right) \). Da \(\gamma \leq 1\), ist auch gut zu sehen, dass gilt: \( \gamma + \ln (n) +\mathcal O\!\left(\frac 1{n}\right) -1 \leq \ln (n) \).

\subsection{b)}
Wenn wir von \(\lim_{n\to\infty} \frac {ln(n)} {H_n} = 1 \) ausgehen und 4. a) berücksichtigen, dann ergibt sich \(\lim_{n\to\infty} \frac {\ln(n)} {\gamma + \ln (n) +\mathcal O\!\left(\frac 1{n}\right) } = 1 \). Da \(\mathcal O\!\left(\frac 1{n}\right)\) verschwindent klein wird und \(\gamma\) eine Konstante ist, haben wir effektiv \(\lim_{n\to\infty} \frac {\ln(n)} {\ln(n)} = 1\).

\end{document}
