\documentclass[a4paper]{scrartcl}
\usepackage[ngerman]{babel}
\usepackage[utf8]{inputenc}
\usepackage[T1]{fontenc}
\usepackage{lmodern}
\usepackage{amssymb}
\usepackage{amsmath}
\usepackage{enumerate}
\usepackage{pgfplots}
\usepackage{scrpage2}\pagestyle{scrheadings}
\usepackage{tikz}
\usetikzlibrary{patterns}

\newcommand{\titleinfo}{Hausaufgaben zum 5. Juli 2012}

\title{\titleinfo}
\author{Elena Noll, Sven-Hendrik Haase, Arne Struck}
\date{\today}
\ihead{EN, SHH, AS}
\chead{\titleinfo}
\ohead{\today}
\setheadsepline{1pt}
\setcounter{secnumdepth}{0}
\newcommand{\qed}{\quad \square}

\begin{document}
\maketitle
\notag
\section{1.}
\subsection{a)}
\begin{align}
&f(x,y,z) = 2x^2 + y^2 + 4z^2 - 2yz - 2x - 6y + 8 \\
&gradf(x,y,z) = (4x - 2, 2y - 2z - 6, 8z - 2y) \\
&gradf(0,0,0)
\end{align}
\begin{align}
4x - 2 &= 0 \\
2y - 2z - 6 &= 0 \\
8z - 2y &= 0 \\
\\
\text{I nach x auflösen} \\
\Rightarrow x &= \frac{1}{2} \\
\\
\text{III nach z auflösen} \\
\Rightarrow z &= \frac{1}{4}y \\
\\
\text{z in II einsetzen und nach y auflösen} \\
\Rightarrow y &= 4 \\
\\
\text{y in III einsetzen} \\
\Rightarrow z &= 1 \\
\\
\Rightarrow (x,y,z) = (\frac{1}{2},4,1)&
\end{align}
\begin{align}
\text{Partielle Ableitungen 1. und 2. Grades bilden} \\
f_x &= 4x - 2 \\
f_{xx} &= 4 \\
f_{xy} &= 0 \\
f_{xz} &= 0 \\
\\
f_y &= 2y - 2z - 6 \\
f_{yx} &= 4 \\
f_{yy} &= 4 \\
f_{yz} &= 4 \\
\\
f_z &= 8z - 2y \\
f_{zx} &= 4 \\
f_{zy} &= 4 \\
f_{zz} &= 4
\end{align}
\begin{align}
\text{Hesse-Matrix aufstellen} \\
\\
\begin{pmatrix}
4 & 0 & 0 \\
0 & 2 & -2 \\
0 & -2 & 8 
\end{pmatrix}
\end{align}
Es handelt sich um eine positive Matrix. \\
\(\Rightarrow\) An der Stelle \((\frac{1}{2}, 4, 1)\) liegt ein strenges lokales Minimum vor.
\subsection{b)}
Die Temperatur steigt am stärksten in Richtung des positiven Gradienten. \\
\\
\begin{align}
gradf(x,y,z) &= (4x - 2, 2y - 2z - 6, 8z - 2y) \\
\text{für} (x,y,z) &= (1,1,1) \\
\Rightarrow gradf &= (2,0,6) \\
\end{align}
\\
In Richtung des Vektors (2, 0, 6) steigt die Temperatur am stärksten.
\newpage %Provisorium, löschen bei Endformatierung
\section{2.}
\subsection{a)}
\[
A=\begin{pmatrix}
i & 1 \\
2 & -i
\end{pmatrix},\ 
B=\begin{pmatrix}
i & -i
\end{pmatrix},\ 
C=\begin{pmatrix}
-i \\
1
\end{pmatrix}
\] \\
\(AB\) existiert nicht, da \(A\) eine \(2\times 2\) Matrix und \(B\) eine \(1\times 2\) Matrix ist. Zur Multiplikation müsste, vorausgesetzt \(A\) und \(B\) sind \(m\times n\) -Matrizen, \(n_A=m_B\) sein.
\begin{align}
AC&=
\begin{pmatrix}
i & 1 \\
2 & -i
\end{pmatrix}\cdot
\begin{pmatrix}
i & -i
\end{pmatrix}=
\begin{pmatrix}
1+i \\
-2i+1
\end{pmatrix} \\
BC&=
\begin{pmatrix}
i & -i
\end{pmatrix}\cdot
\begin{pmatrix}
-i \\
1
\end{pmatrix}=
1-i \\
CB&=
\begin{pmatrix}
-i \\
1
\end{pmatrix}
\cdot
\begin{pmatrix}
i & -i
\end{pmatrix}=
\begin{pmatrix}
1 \\
-i
\end{pmatrix}
\end{align}

\subsection{b)}
\begin{align}
\bar{z}&=a-ib \\
	&=\frac{3+4i}{2-3i} \\
	&=\frac{3+4i}{2-3i}\cdot\frac{2+3i}{2+3i} \\
	&=\frac{6+9i+8i-12}{4+6i-6i-9} \\
	&=\frac{-6+17i}{-5} \\
	&=\frac{6}{5}-\frac{17}{5}i \\ \\
\Rightarrow z&=\frac{6}{5}+\frac{17}{5}i \\ \\
\Rightarrow a&=\frac{6}{5} \\
	b&=\frac{17}{5}
\end{align}

\newpage
\subsection{c)}
\begin{align}
z_1&=\sqrt{2}(\cos\big(\frac{\pi}{4}\big)+i\sin\big(\frac{\pi}{4}\big)) \\
	&=\sqrt{2}\cos\big(\frac{\pi}{4}\big)+i\sqrt{2}\sin\big(\frac{\pi}{4}\big) \\
	&=\sqrt{2}\frac{1}{\sqrt{2}}+i\sqrt{2}\frac{1}{\sqrt{2}} \\
	&=1+i \\ \\
z_2&=-1+i \\ \\
z_3&=z_1\cdot z_2 \\
	&=(1+i)\cdot(-1+i) \\
	&=-1+i-i+i^2 \\
	&=-1-1+0i \\
	&=-2+0i \\ \\
z_4&=\bar{z_2} \\ 
	&=-1-i \\
\end{align}
\begin{center}
\begin{tikzpicture}[scale=2.3][domain=0:2] % Zeichenbereich
% Achsen zeichnen
\draw[->] (-2.5,0) -- (2.5,0) node[right] {$\Re$};
\draw[->] (0,-1.5) -- (0,1.5) node[above] {$\Im$};
%Achsenbeschriftung zeichnen
\foreach \x in {-2,-1,1,2}
\draw (\x,-.1) -- (\x,.1) node[below=10pt] {$\scriptstyle\x$};
\foreach \y in {-1,1}
\draw (-.1,\y) -- (.1,\y) node[left=10pt] {$\scriptstyle\y$};
% Funktionen zeichnen
\draw[thick,blue][->] (0,0) -- (1,1) node[right] {$z_1$};
\draw[thick,orange][->] (0,0) -- (-1,1) node[left] {$z_2$};
\draw[thick,red][->] (0,0) -- (-2,0) node[above] {$z_3$};
\draw[thick,green][->] (0,0) -- (-1,-1) node[left] {$z_4$};
\end{tikzpicture}
\end{center}
\newpage
\subsection{d)}
\subsubsection{(i)}
\(M_1=\Big\{z\in\mathbb{C}:|z-(3+2i)|=2 \Big\}\) \\
\(z-(3+2i)\) beschreibt die Differenz zweier komplexer Zahlen. In der Gaußschen Zahlenebene werden diese als Vektoren aufgefasst, also ist das Ergebnis dieser Subtraktion ein neuer Vektor.
Der Betrag eines Vektors beschreibt seine Länge, welche in diesem Fall 2 beträgt. \\
Man kann einen bestimmten Vektor auch als Teil einer Geraden auffassen, wobei der Vektor noch eine Richtung impliziert. Da die Länge eines Vektors allerdings keine Richtung mehr impliziert, kann \(|z-(3+2i)|\) auch als Länge einer Strecke (begrenzte Gerade) aufgefasst werden.
\subsubsection{(ii)}
\(M_2=\Big\{z\in\mathbb{C}:|z-i|=|z-1| \Big\}\) \\
Analog zu (i) beschreiben die beiden Beträge aus \(M_2\) Längen von Vektoren. Allerdings ist die Gleichung nur unter der Bedingung wahr, dass \(a=b\ |\ z=a+bi\) gilt.
Also beschreibt \(M_2\) die erste Winkelhalbierende der Gaußschen Zahlenebene. \\

\section{3}
\subsection{a)}
\(f(x,y)=-0,1x^2-0,2xy-0,2y^2+47x+48y-600\) \\ \\
Nebenbedingung: \\
\(g(x,y)=x+y-200\) \\
\begin{align}
\begin{pmatrix}
\frac{\partial g}{\partial x}(x,y) & \frac{\partial g}{\partial y}(x,y) 
\end{pmatrix}
&=
\begin{pmatrix}
1 & 1
\end{pmatrix}
\end{align}
\[L(x,y,\lambda)=-0,1x^2-0,2xy-0,2y^2+47x+48y-600+\lambda(x+y-200)\]
\begin{align}
\frac{\partial L}{\partial x}(x,y,\lambda)&=-0,2x-0,2y+47+\lambda \\
\frac{\partial L}{\partial y}(x,y,\lambda)&=-0,2x-0,4y+48+\lambda \\
\frac{\partial L}{\partial \lambda}(x,y,\lambda)&=x+y-200 \\
\end{align}
\newpage
LGS: \\
I:\quad\(-0,2x-0,2y+\lambda = -47\) \\
II: \ \(-0,2x-0,4y+\lambda = -48\) \\
III: \ \quad\quad\quad\quad\quad\(x+y =200\) \\ \\
-5\(\cdot\)I-III:
\begin{align}
x-x+y-y-5\lambda &=235-200 \\
\Leftrightarrow 	-5 \lambda &= 35 \\
\Leftrightarrow 	\lambda &= -7 \\
\end{align}
-5\(\cdot\)II-III:
\begin{align}
x-x+2y-y-5\cdot -7 &= 240-200 \\
\Leftrightarrow	y+35&=40 \\
\Leftrightarrow	y&=5 \\
\end{align}
y in III:
\begin{align}
x+5&=200 \\
\Leftrightarrow x &= 195
\end{align}
\(\Rightarrow\) die kritische Stelle ist \((195,5)\).
\subsection{b)}
\begin{align}
\frac{\partial ^2L}{\partial x^2}(x,y,\lambda)&=-0,2 \\
\frac{\partial ^2L}{\partial y^2}(x,y,\lambda)&=-0,4 \\
\frac{\partial ^2L}{\partial x \partial y}(x,y,\lambda)&=-0,2 \\
\end{align}
\[\bar{H}_f(195,5)=
\begin{pmatrix}
0 & 1 & 1 \\
1 & -0,2 & -0,2 \\
1 & -0,2 & -0,4
\end{pmatrix}
\] \\
\(det\ \bar{H}_f=0,2>0\) \\
\(\Rightarrow (195,5)\) ist ein Maximum.
\newpage
\section{4.}

\end{document}
