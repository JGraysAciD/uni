\documentclass[a4paper]{scrartcl}
\usepackage[ngerman]{babel}
\usepackage[utf8]{inputenc}
\usepackage[T1]{fontenc}
\usepackage{lmodern}
\usepackage{amssymb}
\usepackage{amsmath}
\usepackage{enumerate}
\usepackage{pgfplots}
\usepackage{scrpage2}\pagestyle{scrheadings}
\usepackage{tikz}
\usetikzlibrary{patterns}

\newcommand{\titleinfo}{Hausaufgaben zum 5. Juli 2012}

\title{\titleinfo}
\author{Elena Noll, Sven-Hendrik Haase, Arne Struck}
\date{\today}
\ihead{EN, SHH, AS}
\chead{\titleinfo}
\ohead{\today}
\setheadsepline{1pt}
\setcounter{secnumdepth}{0}
\newcommand{\qed}{\quad \square}

\begin{document}
\maketitle
\notag
\section{1.}
\subsection{a)}
\begin{align}
&f(x,y,z) = 2x^2 + y^2 + 4z^2 - 2yz - 2x - 6y + 8 \\
&gradf(x,y,z) = (4x - 2, 2y - 2z - 6, 8z - 2y) \\
&gradf(0,0,0)
\end{align}
\begin{align}
4x - 2 &= 0 \\
2y - 2z - 6 &= 0 \\
8z - 2y &= 0 \\
\\
\text{I nach x auflösen} \\
\Rightarrow x &= \frac{1}{2} \\
\\
\text{III nach z auflösen} \\
\Rightarrow z &= \frac{1}{4}y \\
\\
\text{z in II einsetzen und nach y auflösen} \\
\Rightarrow y &= 4 \\
\\
\text{y in III einsetzen} \\
\Rightarrow z &= 1 \\
\\
\Rightarrow (x,y,z) = (\frac{1}{2},4,1)&
\end{align}
\begin{align}
\text{Partielle Ableitungen 1. und 2. Grades bilden} \\
f_x &= 4x - 2 \\
f_{xx} &= 4 \\
f_{xy} &= 0 \\
f_{xz} &= 0 \\
\\
f_y &= 2y - 2z - 6 \\
f_{yx} &= 4 \\
f_{yy} &= 4 \\
f_{yz} &= 4 \\
\\
f_z &= 8z - 2y \\
f_{zx} &= 4 \\
f_{zy} &= 4 \\
f_{zz} &= 4
\end{align}
\begin{align}
\text{Hesse-Matrix aufstellen} \\
\\
\begin{pmatrix}
4 & 0 & 0 \\
0 & 2 & -2 \\
0 & -2 & 8 
\end{pmatrix}
\end{align}
Es handelt sich um eine positive Matrix. \\
\(\Rightarrow\) An der Stelle \((\frac{1}{2}, 4, 1)\) liegt ein strenges lokales Minimum vor.
\subsection{b)}
Die Temperatur steigt am stärksten in Richtung des positiven Gradienten. \\
\\
\begin{align}
gradf(x,y,z) &= (4x - 2, 2y - 2z - 6, 8z - 2y) \\
\text{für} (x,y,z) &= (1,1,1) \\
\Rightarrow gradf &= (2,0,6) \\
\end{align}
\\
In Richtung des Vektors (2, 0, 6) steigt die Temperatur am stärksten.
\section{2.}
\subsection{a)}
\subsection{b)}
\subsection{c)}
\subsection{d)}

\section{3.}
\subsection{a)}
\subsection{b)}
\section{4.}

\end{document}

