\documentclass[a4paper]{scrartcl}
\usepackage[ngerman]{babel}
\usepackage[utf8]{inputenc}
\usepackage[T1]{fontenc}
\usepackage{lmodern}
\usepackage{amssymb}
\usepackage{amsmath}
\usepackage{enumerate}
\usepackage{pgfplots}
\usepackage{scrpage2}\pagestyle{scrheadings}
\usepackage{tikz}
\usetikzlibrary{patterns}

\newcommand{\titleinfo}{Hausaufgaben zum 7. Juni 2012}

\title{\titleinfo}
\author{Elena Noll, Sven-Hendrik Haase, Arne Struck}
\date{\today}
\ihead{EN, SHH, AS}
\chead{\titleinfo}
\ohead{\today}
\setheadsepline{1pt}
\setcounter{secnumdepth}{0}
\newcommand{\qed}{\quad \square}

\begin{document}
\maketitle

\section{1.}
\subsection{(i)}
lol
\subsection{(ii)}
\subsection{(iii)}

\section{2.}
\subsection{a)}
\centerline{
\begin{tikzpicture}[scale=1.5]
  \begin{axis}[xmin=0, xmax=20,
    xlabel=$x$,
    ylabel={$f(x) = e^{-x}$}
  ]
    \addplot[smooth,domain=0:20,samples=40]{e^(-x)};
  \end{axis}
\end{tikzpicture}
}
Keine Wendepunkte vorhanden.\\

\centerline{
\begin{tikzpicture}[scale=1.5]
  \begin{axis}[xmin=0, xmax=40,
    xlabel=$x$,
    ylabel={$g(x) = \frac 1 {1+x}$}
  ]
    \addplot[smooth,domain=0:40,samples=40]{1/(1+x)};
  \end{axis}
\end{tikzpicture}
}
Keine Wendepunkte vorhanden.\\

\centerline{
\begin{tikzpicture}[scale=1.5]
  \begin{axis}[xmin=0, xmax=20,
    xlabel=$x$,
    ylabel={$h(x) = \frac 1 {1+x^2}$}
  ]
    \addplot[smooth,domain=0:20, samples=40]{1/(1+x^2)};
    \addplot[color=red,mark=*] coordinates {(1/sqrt(3), 3/4)};
  \end{axis}
\end{tikzpicture}
}
Wendepunkt bei \(x=\frac 1 {\sqrt{3}}\)\\

\subsection{b)}


\subsection{c)}
\centerline{
\begin{tikzpicture}[scale=1.5]
  \begin{axis}[xmin=-1, xmax=1,
    xlabel=$x$,
    ylabel={$h(x) = \frac 1 {\sqrt{1-x^2}}$}
  ]
    \addplot[smooth,domain=-2:2,samples=1000]{1/(sqrt(1-x^2))};
  \end{axis}
\end{tikzpicture}
}

\section{3.}

\section{4.}
\subsection{a)}
\subsection{b)}
\subsection{c)}
\centerline{
\begin{tikzpicture}[scale=1.5]
  \begin{axis}[xmin=0, xmax=24,
    xlabel=$x$,
    ylabel={$f(x) = 9t \cdot e^{-\frac 1 3 t}$}
  ]
    \addplot[smooth,domain=0:24,samples=100]{9*x*e^(-1*x/3)};
    \addplot[color=red,mark=*] coordinates {(6, 7.308105295)};
  \end{axis}
\end{tikzpicture}
}
Der gesuchte Punkt des stärksten Abbaus (Wendepunkt) liegt bei \(x = 6\).

\section{5.}
\subsection{a)}
\subsection{b)}
\subsection{c)}
\subsection{d)}


\end{document}

