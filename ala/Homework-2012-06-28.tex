\documentclass[a4paper]{scrartcl}
\usepackage[ngerman]{babel}
\usepackage[utf8]{inputenc}
\usepackage[T1]{fontenc}
\usepackage{lmodern}
\usepackage{amssymb}
\usepackage{amsmath}
\usepackage{enumerate}
\usepackage{pgfplots}
\usepackage{scrpage2}\pagestyle{scrheadings}
\usepackage{tikz}
\usetikzlibrary{patterns}

\newcommand{\titleinfo}{Hausaufgaben zum 28. Juni 2012}

\title{\titleinfo}
\author{Elena Noll, Sven-Hendrik Haase, Arne Struck}
\date{\today}
\ihead{EN, SHH, AS}
\chead{\titleinfo}
\ohead{\today}
\setheadsepline{1pt}
\setcounter{secnumdepth}{0}
\newcommand{\qed}{\quad \square}

\begin{document}
\maketitle
\notag


\section{1.}
\subsection{(i)}
\begin{align}
f(x,y) &= 2x^2y^2-3xy+4x+2 \\
\frac {\partial f} {\partial x}(x,y) &= 4xy^2 - 3y+4 \\
\frac {\partial f} {\partial y}(x,y) &= 4x^2y - 3x
\end{align}

\subsection{(ii)}
\begin{align}
f(x,y) &= \cos(x^2 y)e^{xy} \\
\frac {\partial f} {\partial x}(x,y) &= -\sin(x^2 y) \cdot 2xy e^{xy} + \cos(x^2 y) \cdot ye^{xy} \\
\frac {\partial f} {\partial y}(x,y) &= -\sin(x^2 y) \cdot ye^{xy} + \cos(x^2 y) \cdot xe^{xy}
\end{align}

\subsection{(iii)}
\begin{align}
f(x,y) &= \frac {\sin(x) + \cos(y)} {x^2+y^2} \\
\frac {\partial f} {\partial x}(x,y) &= \frac {(\sin(x) + \cos(y)) \cdot 2x - \cos(x) \cdot (x^2+y^2)} {(x^2+y^2)^2} \\
\frac {\partial f} {\partial x}(x,y) &= \frac {(\sin(x) + \cos(y)) \cdot 2y + \sin(y) \cdot (x^2+y^2)} {(x^2+y^2)^2}
\end{align}

\subsection{(iv)}
\begin{align}
f(x,y) &= \sqrt{1-x^2-y^2} = (1-x^2-y^2)^{\frac 1 2} \\
\frac {\partial f} {\partial x}(x,y) &= \frac 1 2 (1-x^2-y^2)^{- \frac 1 2} \cdot (-2x) \\
\frac {\partial f} {\partial y}(x,y) &= \frac 1 2 (1-x^2-y^2)^{- \frac 1 2} \cdot (-2y)
\end{align}

\newpage
\section{2.}
\begin{align}
f(x,y) &= x^2y^3+ye^{x^2 y} \\
\frac {\partial f} {\partial x}(x,y) &= 2xy^3 + 2xy^2 e^{x^2 y} \\
\frac {\partial ^2f} {\partial x^2}(x,y) &= 2y^3 + 2y^2e^{x^2 y} + 4x^2y^3e^{x^2 y} \\
\frac {\partial f} {\partial y}(x,y) &= 3x^2y^2 + e^{x^2 y} + x^2 y e^{x^2 y} \\
\frac {\partial ^2f} {\partial y^2}(x,y) &= 6x^2y + 2x^2e^{x^2 y} + x^4ye^{e^2 y} \\
\frac {\partial ^2} {\partial x\partial y}(x,y) &= 6xy^2 + 4xye^{x^2 y} + 2x^3y^2e^{x^2 y} \\
\frac {\partial ^2} {\partial y\partial x}(x,y) &= 6xy^2 + 4xye^{x^2 y} + 2x^3y^2e^{x^2 y} = \frac {\partial ^2} {\partial x\partial y}
\end{align}

\newpage
\section{3.}
\subsection{i)}
\[f(x,y)=-xy+x^2+y^2-y+5\] \\
\begin{align}
\frac{\partial f}{\partial x}(x,y) &= -y + 2x \\
\frac{\partial ^2 f}{\partial x^2}(x,y) &= 2 \\
\frac{\partial f}{\partial y}(x,y) &= -x + 2y -1 \\
\frac{\partial ^2 f}{\partial y^2}(x,y)  &= 2 \\
\frac{\partial ^2 f}{\partial x \partial y}(x,y) &= -1 \\ 
\end{align}
Daraus folgt: \\
I: \ \(0 = -y + 2x \Leftrightarrow 2x = y\) \\
II: \(0 = -x +2y -1 \) \\
\\
I in II:\\ 
\begin{align}
1 &= -x +4x \\
\Leftrightarrow 1 &= 3x \\
\Leftrightarrow \frac{1}{3} &= x \\
\end{align}
Rückeinsetzen in I: \\
\(\frac{2}{3} = y \Rightarrow\) Der kritische Punkt ist \(\Big(\frac{1}{3},\frac{2}{3}\Big)\) \\
\[H_f\Big(\frac{1}{3},\frac{2}{3}\Big) = \begin{pmatrix}
 2 & -1 \\
-1 &  2 
\end{pmatrix}\] \\
\[\triangle_1 = 2\quad \triangle_2 = 2^2-1 = 3\] \\
\(\Rightarrow\ f\) ist positiv definit \(\Leftrightarrow\) der kritische Punkt ist ein Minimum. \\

\subsection{ii)}
\[f(x,y)=xy-y^2x+y+3\] \\
\begin{align}
\frac{\partial f}{\partial x}(x,y) &= y+1 \\
\frac{\partial ^2 f}{\partial x^2}(x,y) &= 0 \\
\frac{\partial f}{\partial y}(x,y) &= x-2y+1 \\
\frac{\partial ^2 f}{\partial y^2}(x,y) &= -2 \\
\frac{\partial ^2 f}{\partial x\partial y}(x,y) &= 1
\end{align}
Daraus folgt: \\
I: \ \(0=y+1\Leftrightarrow y=1\) \\
II: \(0=x-2y+1\) \\
I in II: \\
\\
\[x=1\] \\
\(\Rightarrow\) die kritische Stelle ist (1,1) \\
\[H_f(1,1)=\begin{pmatrix}
0 & 1 \\
1 & -2
\end{pmatrix}\] \\
\[\triangle_1 = 0\quad \triangle_2 = -1\] \\
\(\Rightarrow f\) ist indefinit \(\Leftrightarrow (1,1)\) ist keine Extremstelle. 

\newpage
\subsection{iii)}
\[f(x,y) = x^3+y^3-12x-3y+5\] \\
\begin{align}
\frac{\partial f}{\partial x}(x,y) &= 3x^2-12 \\
\frac{\partial ^2 f}{\partial x^2}(x,y) &= 6x \\
\frac{\partial f}{\partial y}(x,y) &= 3y^2-3 \\
\frac{\partial ^2 f}{\partial y^2}(x,y) &= 6y \\
\frac{\partial ^2 f}{\partial x\partial  y}(x,y) &= 0 \\
\end{align}
\begin{align}
3x^2-12 &= 0 \Leftrightarrow x^2 = 4 \\
3y^2-3 &= 0 \Leftrightarrow y^2 = 1 \\
\end{align}
Daraus folgt: \\
\begin{align}
x_1 = -2\ &\vee\ x_2 = 2 \\
y_1 = -1\ &\vee\ y_2 =1 \\
\end{align}
\(\Rightarrow\) es existieren die kritischen Stellen \((-2,-1)\,,\,(-2,1)\,,\,(2,-1)\,,\,(2,1)\) \\
\newpage
\begin{align}
H_f(x,y) &= \begin{pmatrix}
6x & 0 \\
0 & 6y
\end{pmatrix} \\
H_f(-2,-1) &= \begin{pmatrix}
-12 & 0 \\
0 & -6
\end{pmatrix} \\
\triangle_1 &= -12 \\
\triangle_2 &= -12 \cdot (-6) -0 \\
	&= 72 
\end{align}
\(\Rightarrow H_f(-2,-1)\) ist negativ definit \(\Leftrightarrow (-2,-1)\) ist ein lokales Maximum. \\
\begin{align}
H_f(-2,1) &= \begin{pmatrix}
-12 & 0 \\
0 & 6
\end{pmatrix} \\
\triangle_1 &= -12 \\
\triangle_2 &= -12 \cdot 6 -0 \\
	&= -72 
\end{align}
\(\Rightarrow H_f(-2,1)\) ist indefinit \(\Leftrightarrow (-2,1)\) ist keine Extremstelle. \\
\begin{align}
H_f(2,-1) &= \begin{pmatrix}
12 & 0 \\
0 & -6
\end{pmatrix} \\
\triangle_1 &= 12 \\
\triangle_2 &= 12 \cdot (-6) -0 \\
	&= -72 
\end{align}
\(\Rightarrow H_f(2,-1)\) ist indefinit \(\Leftrightarrow (2,-1)\) ist keine Extremstelle. \\
\begin{align}
H_f(2,1) &= \begin{pmatrix}
12 & 0 \\
0 & 6
\end{pmatrix} \\
\triangle_1 &= 12 \\
\triangle_2 &= 12 \cdot 6 -0 \\
	&= 72 
\end{align}
\(\Rightarrow H_f(2,1)\) ist positiv definit \(\Leftrightarrow (2,1)\) ist ein lokales Minimum. \\


\newpage
\section{4.}
\subsection{a)}

\subsection{b)}

\subsection{c)}

\end{document}