\documentclass[a4paper]{scrartcl}
\usepackage[ngerman]{babel}
\usepackage[utf8]{inputenc}
\usepackage[T1]{fontenc}
\usepackage{lmodern}
\usepackage{amssymb}
\usepackage{amsmath}
\usepackage{enumerate}
\usepackage{scrpage2}\pagestyle{scrheadings}
\usepackage{tikz}
\usetikzlibrary{arrows,automata}

\newcommand{\titleinfo}{Hausaufgaben zum 15.05.2012}
\newcommand{\aufgabe}[1]{\item[\textbf{#1}]}

\title{\titleinfo}
\author{Arne Feil}
\date{\today}
\ihead{FGI1}
\chead{Arne Feil, Sven-Hendrik Haase, Christian Darsow-Fromm}
\ohead{\today}
\setheadsepline{1pt}
\newcommand{\qed}{\quad \square}
\newcommand{\terminal}[1]{\langle #1 \rangle }
\everymath{\displaystyle}
\begin{document}
%\maketitle
\begin{enumerate}

\aufgabe{6.3}
\begin{enumerate}[1.]
 \item

  \textbf{allgemeingültig, erfüllbar}

  $$((A\vee B))\vee (A\Rightarrow A))$$
  Da die Teilformel $(A\Rightarrow A)$ offensichtlich allgemeingültig und mit \textit{oder} zum anderen Teil verknüpft ist, ist die gesamte Formel allgemeingültig.

  $$((A\vee B)\vee\neg B)$$
  $B$ oder $\neg B$ muss immer wahr sein.

  $$(A\vee B)\Rightarrow(A\vee\neg A))$$
  $(A\vee\neg A)$ ist immer wahr, daher auch die ganze Formel.

  \textbf{kontingent, erfüllbar, falsifizierbar}

  $$((A\vee B)\wedge(A\Rightarrow B))$$
  Mit z.B. $A=0$ und $B=0$ ist die Formel unerfüllt und mit $A=1$ und $B=1$ ist sie erfüllt.

  $$((A\vee B)\Rightarrow(\neg A\wedge \neg A))$$
  Mit $A=1$ und $B=1$ ist sie unerfüllt und mit $A=0$ und $B=0$ erfüllt.

  $$((A\vee B)\wedge(B\Leftrightarrow A))$$
  Mit $A=0$ und $B=1$ ist sie unerfüllt und mit $A=1$ und $B=0$ erfüllt.

  \textbf{unerfüllbar, falsifizierbar}
  $$((A\vee B)\wedge(A\Leftrightarrow B)\wedge\neg A)$$
  Wegen $(A\Leftrightarrow B)$ muss $A=B$ sein. Da $A$ (und auch $B$) wegen $(\neg A)$ negativ sein muss, kann die erste Teilformel dann nicht mehr erfüllt sein.

  $$((A\vee B)\Leftrightarrow(\neg A\wedge\neg B))$$
  Wenn $A$ oder $B$ wahr sind, ist der zweite Teil der Formel auf jeden Fall falsch. Wenn $A=B=0$ ist, ist der zweite Teil wahr und damit die Biimplikation immer unerfüllt.

  $$((A\vee B)\wedge(\neg B\Leftrightarrow A)\wedge(A\Leftrightarrow B))$$
  Wegen dem letzten Term muss $A=B$ sein um die Formel zu erfüllen. Der zweite Term ist aber nur erfüllt, wenn $A\neq B$, daher ist die Formel unerfüllbar.

  \item

  \begin{enumerate}[a)]
   \item
   $$((A\Rightarrow B)\Rightarrow(C\Rightarrow D))\equiv ((B\Rightarrow A)\Rightarrow(D\Rightarrow C))$$
   Lässt sich widerlegen mit: $A=0,\ B=0,\ C=0,\ D=1$\\
   Dann ist die linke Seite $0$ und die rechte ist $1$.

   \item
   $$((A\vee(B\vee C))\wedge(D\vee(E\vee F))) \equiv ((A\wedge D)\vee ((B\wedge E)\vee(C\wedge F)))$$
   Lässt sich widerlegen mit: $A=1,\ B=0,\ C=0,\ D=0,\ E=1,\ F=0$\\
   Dann ist die linke Seite $1$ und die rechte Seite ist $0$.

  \end{enumerate}

\end{enumerate}

\aufgabe{6.4}

\begin{enumerate}[1.]
 \item
  $$((A\Leftrightarrow B)\Leftrightarrow C)$$

  Elimination von $\Leftrightarrow$:
  $$(((A\wedge B)\vee(\neg B\wedge\neg A))\Leftrightarrow C)$$
  $$((((A\wedge B)\vee(\neg B\wedge\neg A))\wedge C)\vee(\neg C\wedge\neg((A\wedge B)\vee(\neg B\wedge\neg A))) )$$

  de Morgan:
  $$((((A\wedge B)\vee(\neg B\wedge\neg A))\wedge C) \vee(\neg C\wedge(\neg(A\wedge B)\wedge\neg(\neg B\wedge\neg A))))$$
  nochmal de Morgan:
  $$((((A\wedge B)\vee(\neg B\wedge\neg A))\wedge C) \vee(\neg C\wedge((\neg A\vee \neg B)\wedge(\neg\neg B\vee\neg\neg A))))$$
  doppelte Negation entfernen:
  $$((((A\wedge B)\vee(\neg B\wedge\neg A))\wedge C) \vee(\neg C\wedge((\neg A\vee \neg B)\wedge(B\vee A))))$$

  KNF bilden: % TODO Konjunktionen nach außen treiben.
  $$((((A\wedge B)\vee(\neg B\wedge\neg A))\wedge C) \vee(\neg C\wedge((\neg A\vee \neg B)\wedge(B\vee A))))$$

  DNF bilden: % TODO Disjunktionen nach außen treiben.
  $$((((A\wedge B)\vee(\neg B\wedge\neg A))\wedge C) \vee(\neg C\wedge((\neg A\vee \neg B)\wedge(B\vee A))))$$

\end{enumerate}


\aufgabe{6.5}


\end{enumerate}
\end{document}
