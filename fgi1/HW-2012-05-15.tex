\documentclass[a4paper,10pt]{scrartcl}
\usepackage[ngerman]{babel}
\usepackage[utf8]{inputenc}
\usepackage[T1]{fontenc}
\usepackage{lmodern}
\usepackage{amssymb}
\usepackage{amsmath}
\usepackage{enumerate}
\usepackage{scrpage2}\pagestyle{scrheadings}
\usepackage{tikz}
\usetikzlibrary{arrows,automata}

\newcommand{\titleinfo}{Hausaufgaben zum 15.05.2012}
\newcommand{\aufgabe}[1]{\item[\textbf{#1}]}

\title{\titleinfo}
\author{Arne Feil}
\date{\today}
\ihead{FGI1}
\chead{Arne Feil, Sven-Hendrik Haase, Christian Darsow-Fromm}
\ohead{\today}
\setheadsepline{1pt}
\newcommand{\qed}{\quad \square}
\newcommand{\terminal}[1]{\langle #1 \rangle }
\everymath{\displaystyle}
\begin{document}
%\maketitle
\begin{enumerate}

\aufgabe{6.3}
\begin{enumerate}[1.]
 \item

  \textbf{allgemeingültig, erfüllbar}

  $$((A\vee B))\vee (A\Rightarrow A))$$
  Da die Teilformel $(A\Rightarrow A)$ offensichtlich allgemeingültig und mit \textit{oder} zum anderen Teil verknüpft ist, ist die gesamte Formel allgemeingültig.

  $$((A\vee B)\vee\neg B)$$
  $B$ oder $\neg B$ muss immer wahr sein.

  $$(A\vee B)\Rightarrow(A\vee\neg A))$$
  $(A\vee\neg A)$ ist immer wahr, daher auch die ganze Formel.

  \textbf{kontingent, erfüllbar, falsifizierbar}

  $$((A\vee B)\wedge(A\Rightarrow B))$$
  Mit z.B. $A=0$ und $B=0$ ist die Formel unerfüllt und mit $A=1$ und $B=1$ ist sie erfüllt.

  $$((A\vee B)\Rightarrow(\neg A\wedge \neg A))$$
  Mit $A=1$ und $B=1$ ist sie unerfüllt und mit $A=0$ und $B=0$ erfüllt.

  $$((A\vee B)\wedge(B\Leftrightarrow A))$$
  Mit $A=0$ und $B=1$ ist sie unerfüllt und mit $A=1$ und $B=0$ erfüllt.

  \textbf{unerfüllbar, falsifizierbar}
  $$((A\vee B)\wedge(A\Leftrightarrow B)\wedge\neg A)$$
  Wegen $(A\Leftrightarrow B)$ muss $A=B$ sein. Da $A$ (und auch $B$) wegen $(\neg A)$ negativ sein muss, kann die erste Teilformel dann nicht mehr erfüllt sein.

  $$((A\vee B)\Leftrightarrow(\neg A\wedge\neg B))$$
  Wenn $A$ oder $B$ wahr sind, ist der zweite Teil der Formel auf jeden Fall falsch. Wenn $A=B=0$ ist, ist der zweite Teil wahr und damit die Biimplikation immer unerfüllt.

  $$((A\vee B)\wedge(\neg B\Leftrightarrow A)\wedge(A\Leftrightarrow B))$$
  Wegen dem letzten Term muss $A=B$ sein um die Formel zu erfüllen. Der zweite Term ist aber nur erfüllt, wenn $A\neq B$, daher ist die Formel unerfüllbar.

  \item

  % Das muss noch ausführlicher...
  \begin{enumerate}[a)]
   \item
   $((A\Rightarrow B)\Rightarrow(C\Rightarrow D))\equiv ((B\Rightarrow A)\Rightarrow(D\Rightarrow C))$ \\
   Lässt sich widerlegen mit: $A=0,\ B=0,\ C=0,\ D=1$\\ \\
   Wenn $A$ und $B$ falsch sind und $A$ impliziert $B$, dann ist diese Implikation ($I_1$) wahr. Wenn $C$ falsch, $D$ wahr ist 
   und $C$ impliziert $D$, dann ist diese Implikation ($I_2$) auch wahr. Wenn die $I_1$, $I_2$ wahr sind und $I_1$ impliziert $I_2$, dann
   ist diese Implikation wahr. \\
   Wenn $B$ $A$ impliziert, dann ist diese Implikation ($I_3$) wahr. Die Implikation ($I_4$), $D$ impliziert $C$, ist falsch. $I_3$ impliziert $I_4$ 
   ist also falsch. \\
   $A,B,C$ sind falsch und $D$ ist wahr, ist ein Modell für die linke Seite der Äquivalenzbehauptung, aber nicht für die Rechte. Damit sind die beiden
   Formeln nicht äquivalent.
  
   \item
   $((A\vee(B\vee C))\wedge(D\vee(E\vee F))) \equiv ((A\wedge D)\vee ((B\wedge E)\vee(C\wedge F)))$ \\
   Lässt sich widerlegen mit: $A=1,\ B=0,\ C=0,\ D=0,\ E=1,\ F=0$\\ \\
   Wenn $B$ und $C$ falsch sind, ist die Disjunktion dieser auch falsch, dies disjungiert mit $A$, welches wahr ist, ist dann wahr ($D_{1}$).
   Wenn $E$ wahr und $F$ falsch ist, ist die Disjunktion dieser wahr und disjungt mit $D$, wechles falsch ist, auch wieder wahr ($D_{2}$).
   Konjungiert man $D_{1}$ und $D_{2}$, dann ist diese Konjunktion wahr. \\
   Die Konjunktionen $A$ und $D$, $B$ und $E$, $C$ und $F$ sind alle falsch. Disjungiert man diese drei Konjunktionen, ist diese Disjunktion falsch.
   $B,C,D,F,$ sind falsch und $A,E$ sind wahr ist ein Modell für die linke Seite der Äquivalenzbehauptung, aber nicht für die Rechte. Damit sind die beiden
   Formeln nicht äquivalent.
  \end{enumerate}

\end{enumerate}

\aufgabe{6.4}

\begin{enumerate}[1.]
 \item
  $$((A\Leftrightarrow B)\Leftrightarrow C)$$

  Elimination von $\Leftrightarrow$:
  $$(((A\wedge B)\vee(\neg B\wedge\neg A))\Leftrightarrow C)$$
  $$((((A\wedge B)\vee(\neg B\wedge\neg A))\wedge C)\vee(\neg C\wedge\neg((A\wedge B)\vee(\neg B\wedge\neg A))) )$$

  de Morgan:
  $$((((A\wedge B)\vee(\neg B\wedge\neg A))\wedge C) \vee(\neg C\wedge(\neg(A\wedge B)\wedge\neg(\neg B\wedge\neg A))))$$
  nochmal de Morgan:
  $$((((A\wedge B)\vee(\neg B\wedge\neg A))\wedge C) \vee(\neg C\wedge((\neg A\vee \neg B)\wedge(\neg\neg B\vee\neg\neg A))))$$
  doppelte Negation entfernen:
  $$((((A\wedge B)\vee(\neg B\wedge\neg A))\wedge C) \vee(\neg C\wedge((\neg A\vee \neg B)\wedge(B\vee A))))$$

  KNF bilden: % TODO Konjunktionen nach außen treiben.
  $$((((A\wedge B)\vee(\neg B\wedge\neg A))\wedge C) \vee(\neg C\wedge((\neg A\vee \neg B)\wedge(B\vee A))))$$

  DNF bilden: 
  $$((((A\wedge B)\vee(\neg B\wedge\neg A))\wedge C) \vee(\neg C\wedge((\neg A\vee \neg B)\wedge(B\vee A))))$$
  [Distributivität]
  $$ ((A \wedge B) \wedge C) \vee ((\neg A \wedge \neg B) \wedge C) \vee (((\neg A \wedge (A \vee B)) \vee (\neg B \wedge ( A \vee B))) \wedge \neg C) $$
  $$ ((A \wedge B) \wedge C) \vee ((\neg A \wedge \neg B) \wedge C) \vee (((\neg A \wedge A) \vee (\neg A \wedge B) \vee ((\neg B \wedge A) \vee (\neg B \wedge B)) \wedge \neg C) $$
  [Unerfüllbarkeitsregeln]
  $$ ((A \wedge B) \wedge C) \vee ((\neg A \wedge \neg B) \wedge C) \vee ((( \neg A \wedge B) \vee (\neg B \wedge A)) \wedge \neg C) $$
  [Distributivität]
  $$ ((A \wedge B) \wedge C) \vee ((\neg A \wedge \neg B) \wedge C) \vee ((( \neg A \wedge B) \wedge \neg C) \vee (( \neg B \wedge A) \wedge \neg C) $$
  DNF:
  $$ (A \wedge B \wedge C) \vee (\neg A \wedge \neg B \wedge C) \vee (\neg A \wedge B \wedge \neg C) \vee (\neg B \wedge A \wedge \neg C) $$

	\item{}
	$$((A\Leftrightarrow B)\Leftrightarrow C) \quad \quad D\equiv (A \Leftrightarrow B)$$
	
	\begin{tabular}[t]{c|c|c||c|c}
		$A$ & $B$ & $C$ & $D$ & $D \Leftrightarrow C$ \\\hline
		0&0&0&1&0 \\
		0&0&1&1&1\\
		0&1&0&0&1\\
		0&1&1&0&0\\
		1&0&0&0&1\\
		1&0&1&0&0\\
		1&1&0&1&0\\
		1&1&1&1&1\\
	\end{tabular}
	
	$$ DNF: \quad  D \equiv (A \wedge B) \quad  E \equiv (D \wedge C) \quad F \equiv (\neg A \wedge \neg B) \quad G \equiv (F \wedge C) $$
	$$ H \equiv (\neg A \wedge B) \quad I \equiv (H \wedge \neg C) \quad J \equiv (\neg B \wedge A) \quad K \equiv (J \wedge \neg C) $$
	$$ L \equiv (E \vee G) \quad M \equiv (I \vee K) \quad DNF \equiv (L \vee M)$$
	
	\begin{tabular}[t]{c|c|c||c|c|c|c|c|c|c|c|c|c|c|c|c||c}
		$A$&$B$&$C$&$\neg A$&$\neg B$&$\neg C$&$D$&$E$&$F$&$G$&$H$&$I$&$J$&$K$&$L$&$M$&$DNF$\\\hline
		0&0&0&1&1&1&0&0&1&0&0&0&0&0&0&0&0\\
		0&0&1&1&1&0&0&0&1&1&0&0&0&0&1&0&1\\
		0&1&0&1&0&1&0&0&0&0&1&1&0&0&0&1&1\\
		0&1&1&1&0&0&0&0&0&0&1&0&0&0&0&0&0\\
		1&0&0&0&1&1&0&0&0&0&0&0&1&1&0&1&1\\
		1&0&1&0&1&0&0&0&0&0&0&0&1&0&0&0&0\\
		1&1&0&0&0&1&1&0&0&0&0&0&0&0&0&0&0\\
		1&1&1&0&0&0&1&1&0&0&0&0&0&0&1&0&1\\
	\end{tabular}
\end{enumerate}


\aufgabe{6.5}
\begin{enumerate}[1.]
 \item
 % Begründet sich diese Aufgabe nicht durch die Definition?
 % In 6.5.2 b) steht, dass TAUSCH() angewendet auf eine KNF eine DNF liefert.
 % Verstehe ich die Aufgabe hier nicht oder ist es echt nur das Wiederholen
 % der Definition ein paar Zeilen darüber? ~Sven
 
 \item
 % Wenn F bereits eine KNF ist und dann noch mal KNF() darauf angewendet wird
 % und auf die ganze Suppe dann nochmal TAUSCH() angewendet wird, dann
 % resultiert daraus doch keine DNF mehr, oder?
 %
 % Wenn diese Überlegung Dummsinn ist, wieso irrt sich Peter dann?
 % Wir sehen an 6.5.2 b), dass TAUSCH(KNF) eine DNF ergibt. 
\end{enumerate}

\end{enumerate}
\end{document}
