\documentclass[a4paper]{scrartcl}
\usepackage[ngerman]{babel}
\usepackage[utf8]{inputenc}
\usepackage[T1]{fontenc}
\usepackage{lmodern}
\usepackage{amssymb}
\usepackage{amsmath}
\usepackage{enumerate}
\usepackage{scrpage2}\pagestyle{scrheadings}
\usepackage{tikz}
\usetikzlibrary{arrows,automata}

\newcommand{\titleinfo}{Hausaufgaben zum 01.05.2012}
\newcommand{\aufgabe}[1]{\item[\textbf{#1}]}

\title{\titleinfo}
\author{Arne Feil}
\date{\today}
\ihead{FGI1}
\chead{Arne Feil, Sven-Hendrik Haase, Christian Darsow-Fromm}
\ohead{\today}
\setheadsepline{1pt}
\newcommand{\qed}{\quad \square}
\everymath{\displaystyle}
\begin{document}
%\maketitle
\begin{enumerate}

\aufgabe{4.3}
\begin{enumerate}[1.]
 \item
 \begin{tikzpicture}[->, auto, node distance=3cm, >=latex]
    \tikzstyle{every initial by arrow}=[initial text=,->, >=stealth]

    \node[initial, state] (A) {$q_1$};
    \node[state] (B) [right of=A] {$q_2$};
    \node[accepting, state] (C) [right of=B] {$q_3$};

    % Hat jemand ne Ahnung, was man machen kann, damit die Beschriftung wirklich in einer neuen Zeile landet?
    \draw (A) edge [loop above] node {$a,A|AA$\\
									  $a,B|AB$\\
									  $a,\bot|A\bot$\\
									  $b,A|BA$\\
									  $b,B|BB$\\
									  $b,\bot|B\bot$} (A);
    \draw (A) edge [bend left] node {$a,A|AA$\\
									  $a,B|AB$\\
									  $a,\bot|A\bot$\\
									  $b,A|BA$\\
									  $b,B|BB$\\
									  $b,\bot|B\bot$\\
									  $a,*|*$\\ % ändert nichts am Keller.
									  $b,*|*$} (B);
    \draw (B) edge [loop above] node {$a,A|\epsilon$\\
									  $b,B|\epsilon$} (B);
    \draw (B) edge [bend left] node {$\epsilon,\bot|\epsilon$} (C);
\end{tikzpicture}

Im Zustand $q_1$ wird jeder Buchstabe in den Keller geschrieben.
Der Wechsel zu $q_2$ kann mit jedem Kellerinhalt passieren, entweder mit dem letzten Buchstaben der ersten Worthälfte oder mit dem mittleren Buchstaben.
Bei einer geraden Buchstabenzahl wird er bei dem Übergang in den Keller geschrieben, bei einer ungeraden nicht.
Die Schleife bei $q_2$ läuft so lange, bis der Keller leer ist. Wenn er zu früh leer ist, erreicht er keinen Endzustand.
Das Schlusszeichen $\bot$ wird mit dem letzten $\epsilon$-Übergang zu $q_2$ aus dem Keller geholt. Wenn das alles erfolgreich war, erreicht der Automat dort seinen Endzustand.

\item


\end{enumerate}


\aufgabe{4.4}

\aufgabe{4.5}

\end{enumerate}
\end{document}
