\documentclass[a4paper]{scrartcl}
\usepackage[ngerman]{babel}
\usepackage[utf8]{inputenc}
\usepackage[T1]{fontenc}
\usepackage{lmodern}
\usepackage{amssymb}
\usepackage{amsmath}
\usepackage{enumerate}
\usepackage{scrpage2}\pagestyle{scrheadings}
\usepackage{tikz}
\usetikzlibrary{arrows,automata}

\newcommand{\titleinfo}{Hausaufgaben zum 24.04.2012}
\newcommand{\aufgabe}[1]{\item[\textbf{#1}]}

\title{\titleinfo}
\author{Arne Feil}
\date{\today}
\ihead{FGI1}
\chead{Arne Feil, Sven-Hendrik Haase, Christian Darsow-Fromm}
\ohead{\today}
\setheadsepline{1pt}
\newcommand{\qed}{\quad \square}
\everymath{\displaystyle}
\begin{document}
%\maketitle
\begin{enumerate}

\aufgabe{3.3}
\begin{enumerate}[1.]
 \item
 % Muss hier unbedingt eine \epsilon-Verbindung enthalten sein?
 \begin{tikzpicture}[->, auto, node distance=2.2cm, >=latex]
	\tikzstyle{every initial by arrow}=[initial text=,->, >=stealth]

    \node[initial, state] (A) {$q_1$};
    \node[state] (B) [right of=A] {$q_2$};
    \node[state] (C) [right of=B] {$q_3$};
    \node[accepting, state] (D) [right of=C] {$q_4$};

    \draw (A) edge [bend left] node {0} (B);
    \draw (B) edge [bend left] node {1} (C);
    \draw (C) edge [bend left] node {0} (D);
 \end{tikzpicture}

 \item
 \begin{tikzpicture}[->, auto, node distance=2.2cm, >=latex]
	\tikzstyle{every initial by arrow}=[initial text=,->, >=stealth]

    \node[initial, state] (A) {$q_1$};
    \node[state] (B) [right of=A] {$q_2$};
    \node[state] (C) [right of=B] {$q_3$};
    \node[accepting, state] (D) [right of=C] {$q_4$};

    \draw (A) edge [bend left] node {$\epsilon$} (B);
    \draw (B) edge [loop above] node {0} (B);
    \draw (B) edge [bend left] node {$\epsilon$} (C);
    \draw (C) edge [loop above] node {0,1} (C);
    \draw (C) edge [bend left] node {0} (D);
 \end{tikzpicture}

 \item
 \begin{tikzpicture}[->, auto, node distance=2.2cm, >=latex]
	\tikzstyle{every initial by arrow}=[initial text=,->, >=stealth]

    \node[initial, state] (A) {$q_1$};
    \node[state] (B) [right of=A] {$q_2$};
    \node[state] (C) [right of=B] {$q_3$};
    \node[accepting, state] (D) [right of=C] {$q_4$};

    \draw (A) edge [bend left] node {$\epsilon$} (B);
    \draw (B) edge [loop above] node {0} (B);
    \draw (B) edge [bend left] node {$\epsilon$} (C);
    \draw (C) edge [bend left] node {1} (D);
    \draw (D) edge [loop above] node {0,1} (D);
    \draw (D) edge [bend left] node {0} (C);
 \end{tikzpicture}

\end{enumerate}


\aufgabe{3.4}

\aufgabe{3.5}


\end{enumerate}
\end{document}
