\documentclass[12pt]{article}
\usepackage[utf8]{inputenc}
\usepackage{amssymb,amsmath}
\usepackage{hyperref}
\usepackage{graphicx}
\usepackage{color}
\definecolor{gray}{gray}{.75}
\author{Claas Jaehrling, Sven-Hendrik Haase}
\title{RS1 HA zum 20.01.12}
\date{\today}
\begin{document}
\setcounter{secnumdepth}{0}
\maketitle

\section{Aufgabe 10.1}
\subsection{(a)}
Speicher = 50

\subsection{(b)}
Speicher = 30

\subsection{(c)}
Speicher = 60

\subsection{(d)}
Speicher = 80

\subsection{(e)}
Speicher = 20

\section{Aufgabe 10.2}
\begin{verbatim}
Befehlsindex | Reg1 | Offset
   0000      | 0000 | 0000 0000 0000 0000 0000 0000
 
Befehlsindex | Reg1 | Reg2 | Offset
   0000 0000 | 0000 | 0000 | 0000 0000 0000 0000

Befehlsindex | Offset
   0000 0000 | 0000 0000 0000 0000 0000 0000
\end{verbatim}

\section{Aufgabe 10.3}
\subsection{(a)}
\begin{tabular} {r|ll|ll}
	& dezimal 		& binär 						& rot 		& imm8 \\ \hline
(a) 	& 207 			& 11001111 						& 0000 	& 11001111 \\
(b) 	& 398 			& 110001110 					& xxxx 	& xxxxxxxx \\
(c) 	& 351 			& 101011111 					& xxxx 	& xxxxxxxx \\
(d) 	& 2684354560 	& 10100000000000000000000000000000 	& 0010 	& 00001010 \\
(e) 	& 1212416 		& 100101000000000000000 			& 0100 	& 10010100
\end{tabular} \\\\\\
(b) und (c) sind nicht auf diese Weise darstellbar, da 
(b) das Bitmuster 11000111 aus 110001110 zwar genau \\
8 bit breit ist, die am weitesten links stehende 1 jedoch nicht an einer geraden Stelle im  \\
Muster steht.
(c) das Bitmuster 101011111 breiter als 8 bit ist.
\\\\\\\\\\\\\\\\

\section{Aufgabe 10.4}
\subsection{(a)}
W = (A*B-C)/(D+E*F) \\\\
\begin{tabular} {r|l|l|l|l}
Schritt 	& 0-adress 	& 1-adress 	& 2-adress 	& 3-adress 		\\ \hline
1		& PUSH A 	& LOAD E 	& MOV W,A 	& MUL W,A,B 	\\
2		& PUSH B 	& MUL F 	& MUL W,B 	& SUB W,W,C 	\\
3		& MUL 	& ADD D 	& SUB W,C 	& MUL J,E,F 		\\
4		& PUSH C 	& STORE H	& MOV I,E 	& ADD J,J,D 		\\
5		& SUB 	& LOAD A 	& MUL I,F 	& DIV W,W,J 		\\
6		& PUSH E 	& MUL B	& ADD I,D 	& 		 	\\
7		& PUSH F 	& SUB C 	& DIV W,I 	&  			\\
8		& MUL 	& DIV W 	&  		&  			\\
9		& PUSH D 	& STORE W	&  		& 		 	\\
10		& ADD 	&  		&  		&  			\\
11		& DIV 		& 	 	&  		&  			\\
12		& POP W 	&		&  		& 
\end{tabular}
\\\\
\subsection{(b)}
\begin{tabular} {r|l|l|l|l}
		& 0-adress 	& 1-adress 	& 2-adress 	& 3-adress 	\\ \hline
4-bit		& 0		& 0		& 0		& 3		\\
8-bit 		& 12		& 9		& 7		& 5		\\
16-bit 		& 7		& 9		& 14		& 12		\\ \hline
gesamt	& 208		& 216		& 280		& 244		\\
\end{tabular}
\\\\
Die kompakteste Codierung für dieses Programm hat die 0-Adress-Maschine.
\end{document}
