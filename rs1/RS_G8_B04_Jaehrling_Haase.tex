\documentclass[12pt]{article}
\usepackage[utf8]{inputenc}
\usepackage{amssymb,amsmath}
\usepackage{textcomp} 
\author{Claas Jaehrling, Sven-Hendrik Haase}
\title{RS1 HA zum 25.11.11}
\date{\today}
\begin{document}
\setcounter{secnumdepth}{0}
\maketitle

\section{Aufgabe 4.1}


\section{Aufgabe 4.2}


\section{Aufgabe 4.3}
\subsection{(a)}
\begin{align}
&y = 18 * x\\
&= (2^4) * x + (2^1) * x\\
&= (x << 4) + (x << 1)
\end{align}
\subsection{(b)}
\begin{align}
&y = 14 * x\\
&= (2^4) * x - (2^1) * x\\
&= (x << 4) - (x << 1)
\end{align}
\subsection{(c)}
\begin{align}
&y = -56 * x\\
&= (2^3) * x - (2^6) * x\\
&= (x << 3) - (x << 6)
\end{align}
\subsection{(d)}
\begin{align}
&y = 62 * (x + 2)\\
&= 62 * x + 124\\
&= (2^6) * x - (2^1) * x + 124\\
&= (x << 6) - (x << 1) + 124
\end{align}

\section{Aufgabe 4.4}
\subsection{(a)}
\begin{align}
public int bitNand(int x,int y)\\
&// das, was nicht die Schnittmenge von x und y ist\\
&// (x\& y)  = ~ (x | y)\\
&// = (~ x | ~ y)\\
&return  ~ x | ~ y
\end{align}

\subsection{(b)}
\begin{align}
&public int bitXnor(int x,int y)\\
&// das, was nicht die Schnittmenge von x und y ist\\
&// (x\& y)  = ~ (x | y)\\
&= (~ x | ~ y)\\
&return x | y
\end{align}

\subsection{(c)}
\begin{align}
&public int getByte(x,n)\\
&return
\end{align}
\subsection{(d)}
\begin{align}
&public int rotateLeft(x,n)
\end{align}
\subsection{(e)}
\begin{align}
&public int abs(x)
\end{align}


\section{Aufgabe 4.5}
\subsection{(a)}
\begin{align}
\end{align}
\end{document}
