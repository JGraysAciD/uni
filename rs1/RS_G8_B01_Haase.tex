\documentclass[12pt]{article}
\usepackage[utf8]{inputenc}
\author{Sven-Hendrik Haase}
\title{RS1 HA zum 28.10.11}
\date{\today}
\begin{document}
\maketitle

\section{Aufgabe 1.1}
Auf Ebene \(i = 1\):
\[n(n*k)))\]
Auf Ebene \(i = 2\):
\[n(n(n*k)))\]
Auf Ebene \(i = 3\):
\[n(n(n(n*k)))\]

\section{Aufgabe 1.2}
\subsection{(a)}
\begin{itemize}
\item Das Ablegen von Programmen und Daten im Speicher führt zu schnelleren
Zugriffszeiten.
\item Durch das Selbstmodifizieren werden Tricks wie Selbstoptimierung und
Kompression möglich.
\item Viren können sich selbst modifizieren, um für Virenscanner schwerer sichtbar zu werden.
\end{itemize}
\subsection{(b)}
\begin{itemize}
\item Es könnten Sicherheitsprobleme durch das Selbstmodifizieren auftreten, denn
schließlich läuft danach ein Programm anders als ursprünglich programmiert.
\item Aus dem gleiche Grund könnten Stabilitätsprobleme auftreten.
\item Es wird ein intelligenter Speichermanager benötigt, um zu entscheiden, welche
Programme sinnvoll im Speicher zu bewahren sind und welche nicht, um aktiven Programmen bessere Leistung zu ermöglichen.
\end{itemize}

\section{Aufgabe 1.3}
\subsection{(a)}
\[y = ( a * b ) - ( a * c )\]
\[5 + 1 + 5 = 11 ns\]
\\
\[y = a * ( b - c)\]
\[5 + 1 =  6 ns\]

\subsection{(b)}
Klassisch:
\[y = ( a * x^5 + b * x^4 + c * x^3 + d * x^2 + e * x + f )\]
\[5 + 20 + 1 + 5 + 15 + 1 + 5 + 10 + 1 + 5 + 5 + 1 + 5 + 1  = 80 ns\]
Horner:
\[y = ((((( x * a + b ) * x + c ) * x + d ) * x + e ) * x + f ) \]
\[5 + 1 + 5 + 1 + 5 + 1 + 5 + 1 + 5 + 1 = 30ns\]

\newpage
\section{Aufgabe 1.4}
\subsection{(a)}
Wenn diese Aufzeichnungen in falsche Hände gelangen sollten, könnte dies
katastrophale folgen für das Privatleben und/oder die Karriere des Betroffenen
haben. Dennoch könnte so eine Aufzeichnung auch Vorteile haben: Z.B. als
Beweismaterial bei einem Überfall oder bei Verkehrsunfällen und auch bei
mündlichen Abkommen.
Auch als Studienmaterial für soziale Interaktionen oder als
Selbstverbesserungsmethode für Zeitmanagement.
\\
Sicherlich sollte es eine Methode geben, die Daten im Notfall zu löschen, denn
sonst könnten sich die oben genannten Probleme ergeben. In jedem Fall sollten
die Daten extrem vertraulich behandelt werden.

\subsection{(b)}
Pro Tag:
    \[(5 MB / sec) * (60 * 60 * 24 sec) = 432 GB\]
Pro Jahr:
    \[(5 MB / sec) * (60 * 60 * 24 * 365 sec) = 157,680 TB\]
Für ein Leben:
    \[(5 MB / sec) * (60 * 60 * 24 * 365 * 80 sec) = 12,61 PB\]

\subsection{(c)}
    \[1500 GB * (1,45^x) = 12,61 PB =~ 24 (Jahre)\]

\subsection{(d)}
    \[16 GB * (1,55^x) = 12,61 PB =~ 31 (Jahre)\]
\end{document}
