\documentclass[12pt]{article}
\usepackage[utf8]{inputenc}
\usepackage{amssymb,amsmath}
\author{Claas Jaehrling, Sven-Hendrik Haase}
\title{RS1 HA zum 18.11.11}
\date{\today}
\begin{document}
\maketitle

\section{Aufgabe 3.1}

\subsection{(a)}
\begin{align}
&5385_{10} - 732_{10}\\
&= K_{(K_5385 + 732)}\\
&= K_{(10000-5385 + 732)}\\
&= K_{(4615 + 732)}\\
&= K_{5347}\\
&= 10000-5347\\
&= 4653
\end{align}

\subsection{(b)}
\begin{align}
&732_{10} - 867_{10}\\
&= K_{(K_{732} + 867)}\\
&= K_{(1000-732 + 867)}\\
&= K_{(268 + 867)}\\
&= K_{1135}\\
&= 1000 -1135\\
&= -135
\end{align}

\section{Aufgabe 3.2}

\subsection{(a)}
\begin{align}
&(47,252|3)_{10}\\
&= 47,252 * 10^3\\
&= 4,7252 * 10^4
\end{align}

\subsection{(b)}
\begin{align}
&(-10101,11|-101)_2\\
&= -10101,11 * 2^{-101}\\
&= -1010,111 * 2^{-100}\\
&= -101,0111 * 2^{-11}\\
&= -10,10111 * 2^{-10}\\
&= -1,010111 * 2^{-1}
\end{align}

\subsection{(c)}
\begin{align}
&-0,002DA|C)_{16}\\
&= -0,002DA * 16^C\\
&= -0,02DA * 16^B\\
&= -0,2DA * 16^A\\
&= -2,DA * 16^9
\end{align}

\section{Aufgabe 3.3}

\subsection{(a)}
\begin{align}
&101 1000\\
&1011000,0\\
&Norm:\\
&1,011000 | 1001\\
&1001 = 9\\
&9 + 127 = 136 = 128 + 8 = 10001000\\
&0 | 10001000 | 01100000000000000000000
\end{align}

\subsection{(b)}
\begin{align}
&-10011011,101\\
&-1,0011011101 | 1001\\
&1 | 10001000 | 00110111010000000000000
\end{align}

\section{Aufgabe 3.4}
\begin{align}
&7,516 * 10^6 + 9,9453 * 10^8\\
&= (0,07516 + 9,9453) 10^8\\
&= 10,02046 * 10^8\\
&= 1,002046 * 10^9\\
&\approx 1,002 * 10^9
\end{align}

\section{Aufgabe 3.5}
\begin{align}
&(2,6538 * 10^3) X (3,1415 * 10^5)\\
&= (2,6538 * 3,1415) * 10^8\\
&= 8,3369127 * 10^8\\
&\approx 8,3369 * 10^8
\end{align}

\section{Aufgabe 3.6}
\subsection{(a)}
\begin{verbatim}
0A 44 69 65 73 65 0A 20 4C F6 73 75 6E 67 0A 20 20 62 72 69 6E 67 74 0A
20 20 20 49 68 6E 65 6E 0A 20 20 20 20 28 66 61 73 74 29 0A 20 20 20 20
20 31 35 20 50 75 6E 6B 74 65 21
LF  D  i  e  s  e LF SP  L  ö  s  u  n  g LF SP SP  b  r  i  n  g  t LF
SP SP SP  I  h  n  e  n LF SP SP SP SP  (  f  a  s  t  ) LF SP SP SP SP
SP  1  5 SP  P  u  n  k  t  e  !
\end{verbatim}
\subsection{(b)}
Ich verstehe ich Frage nicht ganz. Wohl auf einem von-Neumann-Rechner? Die Bitzahl der Architektur lässt sich wegen der sowieso in 8-bit vorliegenden Enkodierung nicht herrausfinden. Wegen LF lässt sich vermuten, dass der Text von einem UNIX-ähnlichen Betriebssystem aus enkodiert wurde.

\end{document}
