\documentclass[12pt]{article}
\usepackage[utf8]{inputenc}
\usepackage{amssymb,amsmath}
\usepackage{hyperref}
\author{Claas Jaehrling, Sven-Hendrik Haase}
\title{RS1 HA zum 09.12.11}
\date{\today}
\begin{document}
\setcounter{secnumdepth}{0}
\maketitle
\begin{align}
\end{align}


\section{Aufgabe 6.1}
\subsection{(a)}
\begin{verbatim}
Die Minimaldistanz dieses Codes beträgt 4:
Ändert sich irgendwo ein Bit, so auch das entsprechende Zeilen-
und Spaltenparitätsbit und das Paritätsbit der Spaltenparität.
Insgesamt 4 Bit.
\end{verbatim}

\subsection{(b)}
\begin{verbatim}
Mit diesem 2D-Paritätscode können alle Einbitfehler eindeutig
erkannt und korrigiert werden, weil er quasi "von 3 Seiten"
überprüft wird:
Von der Spalte, der Zeile und der Überprüfung der
Parität der Spaltenparität her.
Einzelne Bitfehler bzw. pro Zeile oder Spalte einmal vorkommende 
Bitfehler führen zu "falschen" Paritätsbits in der Spalte und Zeile,
in der sie auftreten.
Doppelte Bitfehler nebeneinander werden entweder in der Spalte
oder in der Zeile als falsch erkannt; immerhin werden nur
höchstens 2 Blöcke als fehlerhaft bezeichnet.
Bei dreifachen Bitfehlern gilt: in einer Reihe werden sie leicht
erkannt, da die Parität dieser Reihe den falschen Wert liefert,
ansonsten gilt das gleiche wie für 1 oder 2 fehlerhafte Bits.
\end{verbatim}

\subsection{(c)}
\begin{verbatim}
Wenn vier Bits fehlerhaft übertragen werden und diese in
der Tabelle ein Quadrat bilden, also je 2 Zeilen und Spalten
betroffen sind, dann können sie nicht als Fehler erkannt werden.
\end{verbatim}

\subsection{(d)}
\begin{verbatim}
Es handelt sich um Ziehen ohne Zurücklegen und ohne
Beachtung der Reihenfolge, also berechnet sich die Anzahl
der Möglichkeiten als n über k, wobei n hier 49
und k 4 entspricht.
Es gibt also insgesamt 211876 Möglichkeiten.
Die Relation von nicht erkennbaren Vierbitfehlern zur möglichen
Gesamtanzahl ist 9 zu 52.969.
\end{verbatim}

\section{Aufgabe 6.2}
\subsection{(e)}

\subsection{(f)}


\section{Aufgabe 6.3}
\subsection{(a)}

\subsection{(b)}

\end{document}
