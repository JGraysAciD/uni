\documentclass[12pt]{article}
\usepackage[utf8]{inputenc}
\usepackage{amssymb,amsmath}
\usepackage{hyperref}
\usepackage{graphicx}
\usepackage{color}
\definecolor{gray}{gray}{.75}
\author{Claas Jaehrling, Sven-Hendrik Haase}
\title{RS1 HA zum 27.01.12}
\date{\today}
\begin{document}
\setcounter{secnumdepth}{0}
\maketitle

\section{Aufgabe 11.1}
\subsection{(a)}
addl \%ecx, (\%eax) \\
MEM[0x100] = (\%eax) + \%ecx \\
= 0x0000ABBA + 0x00000002 \\
= 0x0000ABBC
\subsection{(b)}
subl \%edx, 4(\%eax) \\
MEM[4(\%eax) = 4(\%eax) - \%edx \\
MEM[0x100 + 0x004] = 0x000000DC - 0x0000000C \\
MEM[0x104] = 0x0000000D0
\subsection{(c)}
imull \$16, (\%eax, \%edx) \\
MEM[\%eax + \%edx] = (\%eax + \%edx) * 16 \\
MEM[0x00000100 + 0x0000000C] = (0x00000100 + 0x0000000C) * 16 \\
MEM[0x10C] = 0x00054321 * 16 \\
= 0x00543210
\subsection{(d)}
incl 8(\%eax) \\
MEM[0x108] = 0x000000EF + 1 \\
= 0x000000F0
\subsection{(e)}
decl \%ecx \\
\%ecx = 0x00000002 - 1 \\
= 0x00000001
\subsection{(f)}
subl \%edx, \%eax \\
\%eax = \%eax - \%edx \\
= 0x00000100 - 0x0000000C \\
= 0x00000FF4

\section{Aufgabe 11.2}
Man könnte einfach den Inhalt eines Registers von sich selber subtrahieren: \\
subl \%eax, \%eax

\section{Aufgabe 11.3}
pushl \%eip \\
popl \%eax

\section{Aufgabe 11.4}

\section{Aufgabe 11.5}

\section{Aufgabe 11.6}

\end{document}
