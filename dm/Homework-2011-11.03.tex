\documentclass{scrartcl}
\usepackage[ngerman]{babel}
\usepackage[utf8]{inputenc}
\usepackage[T1]{fontenc}
\usepackage{lmodern}
\usepackage{amssymb}
%\usepackage{graphicx}
\usepackage{listings}
\usepackage[automark]{scrpage2}
%\usepackage{tikz}
%\usepackage{verbatim}
\pagestyle{scrheadings}
%\usetikzlibrary{calc,trees,positioning,arrows,fit,shapes,calc}

\title{Hausaufgaben zum 03./04. Novemver 2011}
\author{Arne Feil}
\date{01.11.11}

\ihead[]{Arne Feil}
\chead[]{Hausaufgaben zum 03./04. November 2011}
\ohead[]{01.11.11}
\begin{document}
%\maketitle
\textbf{Aufgabe 1}
\begin{description}
\item{a)}
$\displaystyle A(n): \sum_{i=1}^n\frac{1}{i\cdot(i+1)} = 1-\frac{1}{n+1}\\$
\item{b)}
$n=1\\
\frac{1}{2}=1-\frac{1}{3}\\\\
n=2\\
\frac{1}{2}+\frac{1}{6}=\frac{2}{3}=1-\frac{1}{3}\\\\
n=3\\
\frac{1}{2}+\frac{1}{6}+\frac{1}{12}=\frac{3}{4}=1-\frac{1}{4}\\\\
n=4\\
\frac{1}{2}+\frac{1}{6}+\frac{1}{12}+\frac{1}{20}=\frac{4}{5}=1-\frac{1}{5}$
\item{c)}
Der Induktionsanfang wurde schon in b) berechnet.\\
Induktionsannahme ist a).\\
$
\displaystyle\sum_{i=1}^{n+1}\frac{1}{i\cdot(i+1)} = \sum_{i=1}^n\frac{1}{i\cdot(i+1)}+\frac{1}{(n+1)\cdot(n+2)}
$\\
Auf Grund der Induktionannahme folgt:
$\\
\displaystyle{\left(1-\frac{1}{n+1}\right)+\frac{1}{(n+1)(n+2)}=1-\frac{(n+2)-1}{(n+1)(n+2)}\\
= 1-\frac{1}{n+2} \quad \square}
$
\end{description}
\textbf{Aufgabe 2}
\begin{description}
\item{a)}
$n=1\\
2-1=1=1^2\\\\
n=2\\
1+3=4=2^2\\\\
n=3\\
1+3+5=9=3^2\\\\
n=4\\
1+3+5+7=16=4^2$
\item{b)}
$(2\cdot 1-1)+(2\cdot 2-1)+(2\cdot 3-1)+...+(2\cdot n-1)=n^2$\\
Addiert man alle ungeraden Zahlen von 1 bis $n$, dann ist ihre Summe gleich $n^2$.
\item{c)}
Der Induktionsanfang wurde schon in a) berechnet.\\
Die Induktionsannahme steht in 2.\\
$
\displaystyle\sum_{i=1}^{n+1}(2i-1) = \sum_{i=1}^n(2i-1)+(2(n+1)-1)
$\\
Auf Grund der Induktionsannahme folgt:\\
$n^2+(2(n+1)-1\\
=n^2+2n+1=(n+1)^2 \quad \square$
\end{description}
\textbf{Aufgabe 3}
\begin{description}
\item{a)}
Induktionsanfang:\\
$n=7\\
13\cdot 7 = 91 < 2^7 = 128$\\
Induktionsannahme:\\
$13n<2^n$\\
Induktionsschluss:\\
$13(n+1) = 13n + 13\\
\Rightarrow 13n + 13 < 2^n + 13\\
$Da $13n<2^n$ ist auch $13 < 2^n\\
\Rightarrow 13n+13<2^n+13<2^n+2^n=2\cdot 2^n=2^{n+1}\\
13(n+1)<2^{n+1} \quad \square$
\item{b)}
$n^2<2^n\\
n=1 \quad  1\not< 2\\
n=2 \quad 4\not<4\\
n=3 \quad 9\not<8\\
n=4 \quad 16\not<16\\
n=5 \quad 25<32\\
n=6 \quad 36<64$\\
Behauptung die Ungleichung gilt für alle $n\le5$\\
Induktionsanfang: $5^2 = 25 < 2^5 = 32$\\
Induktionsannahme: $n^2<2^n$\\
Induktionsschluss:\\
$(n+1)^{2}=n^2+2n+1<<2^n+3n<n^2+n^2=2n^2<2\cdot2^n=2^{n+1}\\
(n+1)^2<2^{n+1} \quad\square$
\end{description}
\textbf{Aufgabe 4}\\
Induktionsanfang: $n=4\quad 2^4=16 < 4! = 24$\\
Induktionsannahme: $2^n<n!$\\
Induktionsschluss:\\
$2^{n+1}=2\cdot 2^n < 2\cdot n!\\$
Für $n\le4$ gilt $2<(n+1)\\
2\cdot n! < (n+1)\cdot n! = (n+1)! \\
2^{n+1}<(n+1)!$
\end{document}
