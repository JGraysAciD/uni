\documentclass[a4paper]{scrartcl}
\usepackage[ngerman]{babel}
\usepackage[utf8]{inputenc}
\usepackage[T1]{fontenc}
\usepackage{lmodern}
\usepackage{amssymb}
\usepackage{enumerate}
\usepackage{scrpage2}\pagestyle{scrheadings}

\newcommand{\titleinfo}{Hausaufgaben zum 10./11. November 2011}

\title{\titleinfo}
\author{Elena Noll, Sven-Hendrik Haase, Arne Feil}
\date{\today}
\ihead{EN, SHH, AF}
\chead{\titleinfo}
\ohead{\today}
\setheadsepline{1pt}
\newcommand{\qed}{\quad \square}

\begin{document}
\maketitle
\begin{enumerate}[1.]
\item
\begin{enumerate}[a)]
\item
\begin{enumerate}[(i)]
\item $177 \not\equiv 18 (mod 5)$ da $5 \nmid 159$
\item $177 \equiv 18 (mod 5)$ da $5 \mid 195$
\item $-89\not\equiv -12 (mod 6)$ da $6 \nmid -77$
\item $-123 \equiv 33 (mod 13)$ da $13 \mid -156$
\item $39 \equiv -1 (mod 40)$ da $40 \mid 40$
\item $77 \equiv 0 (mod 11)$ da $11 \mid 77$
\item $2^{51} \not\equiv 51 (mod 2)$ da $2 \nmid 2^{51}-51$\\
\end{enumerate}
\item
$ggT(7293,378) \\
7293 = 19 \cdot 378 + 111 \\
378 = 3 \cdot 111 + 45 \\
111 = 2 \cdot 45 + 21 \\
45 = 2 \cdot 21 + 3 \\
21 = 7 \cdot 3 + 0 \\ \\
ggT(7293,378)=3$\\
\item
$\lceil\sqrt{7}\rceil = 3 \quad \lfloor\sqrt{7}\rfloor = 2 \quad \lceil7.1\rceil = 8 \quad \lfloor7.1\rfloor = 7 \\
\lceil-7.1\rceil = -7 \quad \lfloor-7.1\rfloor = -8 \quad \lceil-7\rceil = -7 \lfloor-7\rfloor = -7$\\
\end{enumerate}
\item
\begin{enumerate}
\item[(2)]
Aus $b_1 \mid a_1$ und $b_2 \mid a_2$ folgt $b_1 \cdot b_2 \mid a_1 \cdot a_2$ \\
\begin{enumerate}
\item[I] $a_1 = b_1 \cdot c_1$
\item[II] $a_2 = b_2 \cdot c_2$\\
\item[I*II] $ a_1 \cdot a_2 = b_1 \cdot b_2 \cdot c_1 \cdot c_2 \quad c_1 \cdot c_2 = c_3\\
a_1 \cdot a_2 = b_1 \cdot b_2 \cdot c_3 \qed$
\end{enumerate}
\item[(3)]
Aus $ c \cdot b \mid c \cdot a$ (für $c \ne	0$) folgt $b \cdot a$\\
$c \cdot a = c \cdot b \cdot c \quad | \div c\\
a = b \cdot c \qed$
\item[(4)]
Aus $b \mid a_1$ und $b \mid a_2$ folgt $b \mid c_1 \cdot a_1 + c_2 \cdot a_2$ für beliebige ganze Zahlen $c_1$ und $c_2$.\\
\begin{enumerate}
\item[I] $a_1 = b \cdot d_1 \quad | \cdot c_1\\
a_1 \cdot c_1 = b \cdot d_1 \cdot c_1$
\item[II] $a_2 = b \cdot d_2 \quad | \cdot c_2\\
a_2 \cdot c_2 = b \cdot d_2 \cdot c_2$
\item[I+II] $a_1 \cdot c_1 + a_2 \cdot c_2 = b \cdot d_1 \cdot c_1 + b \cdot d_2 \cdot c_2\\
a_1 \cdot c_1 + a_2 \cdot c_2 = b(c_1 \cdot d_1 + c_2 \cdot d_2) \quad | c_1 \cdot d_1 + c_2 \cdot d_2 = c_3\\
a_1 \cdot c_1 + a_2 \cdot c_2 = b \cdot c_3 \qed$ 
\end{enumerate}
%Ende von Aufgabe 2
\end{enumerate}
\item
\begin{enumerate}[a)]
\item
$3 \mid (n^3+2n)$\\
Induktionsannahme: $n^3+2n = 3 \cdot c$\\
Induktionsanfang: $n=0\quad 0^3+2\cdot 0=3\cdot c$ für $c=0$\\
Induktionsschritt: $(n+1)^3+2(n+1)=n^3+3n^2+3n+1+2n+2=n^3+2n+3n^2+3n+3$\\
Durch unsere Induktionannahme wissen wir, dass $3\mid n^3+2n$ und es ist klar das $3\mid 3n^2+3n+3$.\\
Daraus folgt: $3\mid (n+1)^3+2(n+1) \qed$
\item
Ein Schachbrett der Größe $2^1 \times 2^1$ hat vier Felder der Größe $1 \times 1$. Überdeckt man dieses Schachbrett mit einem L-Stück, welches
drei Felder der Größe $1 \times 1$ hat, so bleibt ein Feld frei (Induktionsanfang).\\
Daraus folgt (Induktionsannahme): $2^n \times 2^n = (2^n \times 2^n -1) + 1$ Die Klammern dienen nur der Verdeutlichung.\\
Induktionsschritt: $2^{n+1} \times 2^{n+1} = 2\cdot 2^n \times 2 \cdot 2^n$ \\
Durch verwenden der Induktionsannahme folgt:\\
$2\cdot 2^n \times 2 \cdot 2^n -1 +1 = 2^{n+1} \times 2^{n+1} -1 +1 \qed$
%Ende von Aufgabe 3
\end{enumerate}
\newpage
\item
\begin{enumerate}[a)]
\item
Angenommen, für $n,m,k,l \in \mathbb{Q}$ gelte $n,m \not= k,l$ und $g(n,m)=g(k,l)$, dann wäre $g$ nicht injektiv.\\
\begin{enumerate}
\item[I] $nm^2=kl^2$
\item[II] $nm^2-3n=kl^2-3k$
\item[II-I] $3n=3k\\
n=k$\\
\item[III] $(n^2-2)m = (k^2-2)l \quad n=k \\
(k^2-2)m=(k^2-2)l\\
m=l$
\end{enumerate}
Daraus folgt $n,m = k,l$. Das steht im Widerspruch zur Annahme. $g$ ist also injektiv. $\qed$
\item
Annahme: $h$ ist surjektiv. D.h. es gibt $z \in \mathbb{Z}$ für das gilt $h(z)=(x,x)$ mit $x \in \mathbb{Z}$\\
$x=0\\
z+2 = 0\quad $ und $z-1=0\\
z=-2\quad $ und $z=1$ Widerspruch zur Annahme $\qed$.
%Ende von Aufgabe 4
\end{enumerate}
%Ende von allen Aufgaben
\end{enumerate} 
\end{document}
