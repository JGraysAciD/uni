\documentclass[a4paper]{scrartcl}
\usepackage[ngerman]{babel}
\usepackage[utf8]{inputenc}
\usepackage[T1]{fontenc}
\usepackage{lmodern}
\usepackage{amssymb}
\usepackage{amsmath}
\usepackage{enumerate}
\usepackage{scrpage2}\pagestyle{scrheadings}
\usepackage{tikz}

\newcommand{\titleinfo}{Hausaufgaben zum 12./13. Januar 2012}

\title{\titleinfo}
\author{Elena Noll, Sven-Hendrik Haase, Arne Feil}
\date{\today}
\ihead{EN, SHH, AF}
\chead{\titleinfo}
\ohead{\today}
\setheadsepline{1pt}
\newcommand{\qed}{\quad \square}

\begin{document}
\maketitle

\begin{enumerate}
\item[\textbf{1.}]
\begin{enumerate}[a)]
\item
$H_1$ ist keine Untergruppe von $G$, da $H_1$ nicht abgeschlossen ist:\\
$2\cdot 10~mod 13 = 7$\\
$H_2$ ist auch keine Untergruppe von $G$, da auch diese nicht abgeschlossen ist:\\
$2\cdot 8~mod 13 = 3$\\
$H_3$ ist eine Untergruppe. Sie hat ein neutrales Element 1:\\
$1\cdot 12 = 12 = 12 \cdot 1$\\
Und jedes Element hat auch ein Inverses:\\
$1\cdot 1~mod 13 = 1\\
12\cdot 12~mod 13 = 1$
\item 
$H = \{1{,}3{,}9\}$\\
$1H = \{1{,}3{,}9\}$\\
$2H = \{2,5,6\} \\
 3H = \{1{,}3{,}9\} \\
 4H = \{4,10,12\} \\
 5H = \{2,5,6\} \\
 6H = \{2,5,6\} \\
 7H = \{7,8,11\} \\
 8H = \{7,8,11\} \\
 9H = \{1{,}3{,}9\} \\
10H = \{4,10,12\} \\
11H = \{7,8,11\} \\
12H = \{4,10,12\} $
\end{enumerate}
\item[\textbf{2.}]
\begin{enumerate}[a)]
\item
Linksnebenklassen:\\
$idH = \{ id , (1,2) \} \\
(1,2)H = \{ id , (1,2)\} \\
(1,3)H = \{ (1,3) , (1,2,3) \} \\
(2,3)H = \{ (2,3) , (1,3,2) \} \\
(1,2,3)H = \{ (1,3) , (1,2,3) \} \\
(1,3,2)H = \{ (2,3) , (1,3,2) \}$ \\ \\
Rechtsnebenklassen:\\
$Hid = \{ id , (1,2) \} \\
H(1,2) = \{ id , (1,2) \} \\
H(1,3) = \{ (1,3) , (1,3,2) \} \\
H(2,3) = \{ (2,3) , (2,3,1) \} \\
H(1,2,3) = \{ (1,3) , (1,3,2) \} \\
H(1,3,2) = \{ (2,3) , (2,3,1) \} $
\item
$S_6$ ist eine zyklische Untergruppe und somit abelsch. Daher gilt: 
$gH = Hg$.
\item 
\end{enumerate}
%Ende aller Aufgaben
\end{enumerate}
\end{document}

