\documentclass[a4paper]{scrartcl}
\usepackage[ngerman]{babel}
\usepackage[utf8]{inputenc}
\usepackage[T1]{fontenc}
\usepackage{lmodern}
\usepackage{amssymb}
\usepackage{amsmath}
\usepackage{enumerate}
\usepackage{scrpage2}\pagestyle{scrheadings}
\usepackage{tikz}

\newcommand{\titleinfo}{Hausaufgaben zum 12./13. Januar 2012}

\title{\titleinfo}
\author{Elena Noll, Sven-Hendrik Haase, Arne Feil}
\date{\today}
\ihead{EN, SHH, AF}
\chead{\titleinfo}
\ohead{\today}
\setheadsepline{1pt}
\newcommand{\qed}{\quad \square}

\begin{document}
\maketitle

\begin{enumerate}
\item[\textbf{1.}]
\begin{enumerate}[a)]
\item
$H_1$ ist keine Untergruppe von $G$, da $H_1$ nicht abgeschlossen ist:\\
$2\cdot 10~mod 13 = 7$\\
$H_2$ ist auch keine Untergruppe von $G$, da auch diese nicht abgeschlossen ist:\\
$2\cdot 8~mod 13 = 3$\\
$H_3$ ist eine Untergruppe. Sie hat ein neutrales Element 1:\\
$1\cdot 12 = 12 = 12 \cdot 1$\\
Und jedes Element hat auch ein Inverses:\\
$1\cdot 1~mod 13 = 1\\
12\cdot 12~mod 13 = 1$
\item 
$H = \{1{,}3{,}9\}$\\
$1H = \{1{,}3{,}9\}$\\
$2H = \{2,5,6\} \\
 3H = \{1{,}3{,}9\} \\
 4H = \{4,10,12\} \\
 5H = \{2,5,6\} \\
 6H = \{2,5,6\} \\
 7H = \{7,8,11\} \\
 8H = \{7,8,11\} \\
 9H = \{1{,}3{,}9\} \\
10H = \{4,10,12\} \\
11H = \{7,8,11\} \\
12H = \{4,10,12\} $
\end{enumerate}
\item[\textbf{2.}]
\begin{enumerate}[a)]
\item
Linksnebenklassen:\\
$idH = \{ id , (1,2) \} \\
(1,2)H = \{ id , (1,2)\} \\
(1,3)H = \{ (1,3) , (1,2,3) \} \\
(2,3)H = \{ (2,3) , (1,3,2) \} \\
(1,2,3)H = \{ (1,3) , (1,2,3) \} \\
(1,3,2)H = \{ (2,3) , (1,3,2) \}$ \\ \\
Rechtsnebenklassen:\\
$Hid = \{ id , (1,2) \} \\
H(1,2) = \{ id , (1,2) \} \\
H(1,3) = \{ (1,3) , (1,3,2) \} \\
H(2,3) = \{ (2,3) , (2,3,1) \} \\
H(1,2,3) = \{ (1,3) , (1,3,2) \} \\
H(1,3,2) = \{ (2,3) , (2,3,1) \} $
\item
$S_6$ ist eine zyklische Untergruppe und somit abelsch. Daher gilt: 
$gH = Hg$.
\item 
\end{enumerate}
\item[\textbf{3.}]
\begin{enumerate}[a)]

\item
\begin{align}
(a+b)x &= 2x^3+5x^2-4x+4\\
(a*b)x &= 8x^5+4x^4-12x^3+8x^2-2x^4-x^3+3x^2-2x+4x^3+2x^2-6x+4\\
       &= 8x^5+2x^4-9x^3+15x^2-8x+4
\end{align}

\item
\begin{align}
-2x^7+6x^7-18x^7+9x^7-7x^7+40x^7+6x^7+2x^7 =36x^7
\end{align}

\item
\begin{align}
(a+b)x &= 3x^4+4x^3+3x^2+3x+0\\
(a*b)x &= 2x^2+4x^0+2x^3+1x^1+2x^4+1x^2+4x^0+3x^3+4x^1+1x^4+2x^2+1\\
       &= 3x^4+5x^3+5x^2+5x+9
\end{align}
\end{enumerate}

\item[\textbf{4.}]
\begin{enumerate}[a)]
\item
\begin{align}
 &(x^5 +& 2x^4  +& 3x^3 +&  x^2  +& 4x   +& 2) &: (x^2 + 4x + 3) = s.u.\\
-&(x^5 +& 4x^4  +& 3x^3) \\
\cline{1-4}
 &      &(-2x^4 +&       &  x^2) \\
 &     -&(-2x^4 -& 8x^3 -& 6x^2) \\
\cline{3-5}
 &      &        &(8x^3 +& 7x^2  +&  4x) \\
 &      &       -&(8x^3 +&32x^2  +& 24x) \\
\cline{4-6}
 &      &        &       &(-25x^2-& 20x  +& 2) \\
 &      &        &      -&(-25x^2-& 100x -& 75) \\
\cline{5-7}
 &      &        &       &        &  80x +& 77
\end{align}
\[=x^3-2x^2+8x-25, Rest:80x+77\]

\item
\[6x^5  + 7x^4 - 7x^3 - 22x^2 - 25x - 15 = (2x+1)(3x^4 + 2x^3 - 6x^2 - 6x - 9)+3x^3-4x^2-x-6\]
\[3x^4 + 2x^3 - 6x^2 - 6x - 9 = (x+2)(3x^3-4x^2-x-6)+3x^2+2x+3\]
\[3x^3-4x^2-x-6 = (x-2)(3x^2+2x+3)+0\]
\(d(x) = (3x^2+2x+3)\) ist ein größter gemeinsamer Teiler. Der normierte
größte gemeinsame Teiler ist \(\frac 1 3 d(x) = x^2+\frac 2 3 x + 1\).

\end{enumerate}
%Ende aller Aufgaben
\end{enumerate}
\end{document}

