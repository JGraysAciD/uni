\documentclass[a4paper]{scrartcl}
\usepackage[ngerman]{babel}
\usepackage[utf8]{inputenc}
\usepackage[T1]{fontenc}
\usepackage{lmodern}
\usepackage{amssymb}
\usepackage{enumerate}
\usepackage{scrpage2}\pagestyle{scrheadings}

\newcommand{\titleinfo}{Hausaufgaben zum 17./18. November 2011}

\title{\titleinfo}
\author{Elena Noll, Sven-Hendrik Haase, Arne Feil}
\date{\today}
\ihead{EN, SHH, AF}
\chead{\titleinfo}
\ohead{\today}
\setheadsepline{1pt}
\newcommand{\qed}{\quad \square}

\begin{document}
\maketitle
\begin{enumerate}[1.]
\item
\begin{enumerate}[a)]
\item
%Es gibt für die Abbildung $g:X \rightarrow Y$ genau $7^5=16807$ Möglichkeiten $X$ 
%auf $Y$ abzubilden. Da für wir 5 mal aus einer 7-elementigen Menge ziehen und $g$ 
%und für ein mehrere Elemente aus $X$ das selbe Element aus $Y$ haben dürfen. 
%und es nicht unwichtig ist, ob ${1,2}$ gezogen wurden oder ${2,1}$.
%Wenn $g$ injektiv sein soll, ist es als würde man 5 mal aus einer 7-elementigen 
%Menge ziehen ohne wieder zurückzulegen. 
\item
${49 \choose 6}=13 983 816$
\item
$|M|=1000 \quad  997 \leq k \leq 1000 \\
{1000 \choose 997}+{1000 \choose 998}+{1000 \choose 999}+1=166 217 951$
\end{enumerate}
\item
\begin{enumerate}[a)]
\item
Der Koeffizient von $x^5y^{11}$ in $(x+y)^{16}$ lautet: ${16 \choose 5,11}= 
\frac{16!}{5!11!}=4368$ \\
Der Koeffizient von $x^3y^5z^2$ in $(x+y+z)^{10}$ lautet: ${10 \choose 3,5,2}= 
\frac{10!}{3!5!2!}=2520$
\item
$C_1 A P_1 P_2 U C_2 C_3 I N O$
$\frac{10!}{3!\cdot 2!}=302400$\\
$MANGOLAS_1S_2I
\frac{10!}{2!}=1 814 400$\\
$SE_1LTE_2R_1WAS_1S_2E_3R_2
\frac{12!}{2!\cdot 3!\cdot 2!}=19 958 400$
\item
Das Füllen der Kiste betrachten wir als Ziehen mit Zurücklegen ungeordnet von 
$k=6$ Flaschen aus $n=10$ Sorten.\\
${k+n-1 \choose k}={15 \choose 6}=5005$
\end{enumerate}
\item
Induktionanfang:\\
$n=3\\
{3 \choose 0}=1={4 \choose 4}$\\
Induktionannahme:\\
$\displaystyle \sum_{i=3}^{n}{i \choose i-3}={n+1 \choose 4}$\\
Induktionsschritt:\\
$\displaystyle \sum_{i=3}^{n+1}{i \choose i-3}=\sum_{i=3}^{n}{i \choose i-3}+
{n+1 \choose n-2}$\\
Durch Verwenden der Induktionsannahme erhalten wir:\\
$\displaystyle {n+1 \choose 4}+{n+1 \choose n-2}={n+1 \choose 4}+{n+1 \choose 
n+1-(n-2)}={n+1 \choose 4}+{n+1 \choose 3}={n+2 \choose 4} \qed$
\item
\begin{enumerate}[a)]
\item
Es sei $S = \{k \in \mathbb{N}: 1\leq k \leq 2000\}$. Es soll die Anzahl derjenigen 
$k \in S$ bestimmt werden, die weder durch 3, noch durch 5, noch durch 7 teilbar 
sind.\\
Es sei $A_1 = \{k \in S:3|k\},~A_2=\{k \in S :5|k\},~A_3=\{k \in S:7:k\}$. Es folgt 
$N = |S| = 2000$ sowie $|A_1|=\lfloor \frac{2000}{3} \rfloor = 666,~A_2=
\frac{2000}{5}=400,~A_3=\lfloor \frac{2000}{7} \rfloor=285$\\
Ferner haben wir $A_1 \cap A_2 = \{k \in S :15|k\},~A_2 \cap A_3=\{k \in S :35|k\},~
A_1 \cap A_3=\{k \in S:21| k\}$ und $A_1 \cap A_2 \cap A_3 = \{k \in S : 105 |k\}$. 
$|A_1 \cap A_2| = \lfloor \frac{2000}{15} \rfloor=133,~|A_2 \cap A_3|=\lfloor 
\frac{2000}{35}\rfloor=57,~|A_1 \cap A_3|=\lfloor \frac{2000}{21} \rfloor = 95,~
|A_1 \cap A_2 \cap A_3|=\lfloor \frac{2000}{105} \rfloor=19$.\\
Insgesamt erhält man $|S \backslash (A_1 \cup A_2 \cup A_3)| = 2000 - (666+400+285) + 133 
+ 57 +95 - 19 = 915$.
\item
$S= \{k \in \mathbb{N}: 1 \leq k \leq 1000\}$. Es soll die Anzahl derjenigen 
$k \in S$ bestimmt werden, die weder durch 3, noch durch 5, noch durch 7, noch 
durch 11 teilbar sind.\\
$A_1=\{k \in S:3|k\} \\
 A_2=\{k \in S:5|k\} \\
 A_3=\{k \in S:7|k\} \\
 A_4=\{k \in S:11|k\} \\
|A_1| = \lfloor \frac{1000}{3} \rfloor=333 \quad 
|A_2| = \frac{1000}{5} =200 \quad
|A_3| = \lfloor \frac{1000}{7} \rfloor = 142 \\
|A_4| = \lfloor \frac{1000}{11} \rfloor = 90 \\ \\
A_1 \cap A_2 = \{k \in S : 15|k\} \quad
A_2 \cap A_3 = \{k \in S : 35|k\} \quad
A_3 \cap A_4 = \{k \in S : 77|k\} \\
A_1 \cap A_3 = \{k \in S : 21|k\} \quad
A_2 \cap A_4 = \{k \in S : 55|k\} \quad
A_1 \cap A_4 = \{k \in S : 33|k\} \\
A_1 \cap A_2 \cap A_3 \cap A_4 = \{k \in S : 1155|k\} \\ \\
|A_1 \cap A_2| = \lfloor \frac{1000}{15} \rfloor = 66 \quad
|A_2 \cap A_3| = \lfloor \frac{1000}{35} \rfloor = 28 \quad
|A_3 \cap A_4| = \lfloor \frac{1000}{77} \rfloor = 12 \\
|A_1 \cap A_3| = \lfloor \frac{1000}{21} \rfloor = 47 \quad
|A_2 \cap A_4| = \lfloor \frac{1000}{55} \rfloor = 18 \quad
|A_1 \cap A_4| = \lfloor \frac{1000}{33} \rfloor = 30 \\
|A_1 \cap A_2 \cap A_3 \cap A_4 = \lfloor \frac{1000}{1155} = 0 \\ \\
|S \backslash (A_1 \cup A_2 \cup A_3 \cup A_4)| = 1000 - (333+200+142+90) + 66
+ 28 + 12 + 47 + 18 + 30 - 0 = 438
$
\end{enumerate}
% Ende aller Aufgaben
\end{enumerate}
\end{document}
