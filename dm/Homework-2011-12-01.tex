\documentclass[a4paper]{scrartcl}
\usepackage[ngerman]{babel}
\usepackage[utf8]{inputenc}
\usepackage[T1]{fontenc}
\usepackage{lmodern}
\usepackage{amssymb}
\usepackage{amsmath}
\usepackage{enumerate}
\usepackage{scrpage2}\pagestyle{scrheadings}
\usepackage{tikz}

\newcommand{\titleinfo}{Hausaufgaben zum 1./2. Dezember 2011}

\title{\titleinfo}
\author{Elena Noll, Sven-Hendrik Haase, Arne Feil}
\date{\today}
\ihead{EN, SHH, AF}
\chead{\titleinfo}
\ohead{\today}
\setheadsepline{1pt}
\newcommand{\qed}{\quad \square}

\begin{document}
\maketitle

\begin{enumerate}
\item[\textbf{1.}]
\begin{enumerate}[a)]
\item
$AB:\begin{pmatrix}
7 & 5 & -2 \\
2 & 1 & -1 \\
30 & 17 & 5 \\
3 & 2 & 5
\end{pmatrix}$

BA: geht nicht

AC: geht nicht

$AD:\begin{pmatrix}
2 \\
4 \\
26 \\
-4
\end{pmatrix}$

AA: geht nicht

$BB:\begin{pmatrix}
10 & 5 & 1 \\
5 & 4 & -1 \\
4 & 2 & 1
\end{pmatrix}$

$CD:\begin{pmatrix}
12
\end{pmatrix}$

$DC:\begin{pmatrix}
2 & 4 & -4 \\
3 & 6 & -6 \\
-2 & -4 & 4
\end{pmatrix}$

\item
Element, das in AB an dritter Zeile und zweiter Spalte steht: 15

Vierte Spalte von AB: $\begin{pmatrix}
13 \\
8 \\
3 \\
23
\end{pmatrix}$
\end{enumerate}

\item[\textbf{2.}]
\begin{enumerate}[a)]
\item
\begin{align*}
B_1 + B_2 =
\begin{pmatrix}
1 & 2 \\
3 & 6
\end{pmatrix}
+
\begin{pmatrix}
1 & -2 \\
3 & 2
\end{pmatrix}
=
\begin{pmatrix}
2 & 0 \\
6 & 8
\end{pmatrix}
\end{align*}

\begin{align*}
A*(B_1 + B_2) =
\begin{pmatrix}
5 & 7 \\
9 & -1 \\
8 & 2
\end{pmatrix}
*
\begin{pmatrix}
2 & 0 \\
6 & 8
\end{pmatrix}
=
\begin{pmatrix}
52 & 56 \\
2 & -8 \\
28 & 16
\end{pmatrix}
\end{align*}

\begin{align*}
A*B_1 =
\begin{pmatrix}
5 & 7 \\
9 & -1 \\
8 & 2
\end{pmatrix}
*
\begin{pmatrix}
1 & 2 \\
3 & 6
\end{pmatrix}
=
\begin{pmatrix}
26 & 52 \\
6 & 12 \\
14 & 28
\end{pmatrix}
\end{align*}

\begin{align*}
A*B_2 =
\begin{pmatrix}
5 & 7 \\
9 & -1 \\
8 & 2
\end{pmatrix}
*
\begin{pmatrix}
5 & 7 \\
9 & -1 \\
8 & 2
\end{pmatrix}
=
\begin{pmatrix}
26 & 4 \\
6 & -20 \\
14 & -12
\end{pmatrix}
\end{align*}

\begin{align*}
A*B_1+A*B_2 =
\begin{pmatrix}
26 & 52 \\
6 & 12 \\
14 & 28
\end{pmatrix}
+
\begin{pmatrix}
26 & 4 \\
6 & -20 \\
14 & -12
\end{pmatrix}
=
\begin{pmatrix}
52 & 56 \\
12 & -8 \\
28 & 16
\end{pmatrix}
\end{align*}

\begin{align*}
\Rightarrow A*(B_1 + B_2) = A*B_1+A*B_2
\end{align*}

\item
\begin{align*}
A*B =
\begin{pmatrix}
1 & 3 \\
2 & 6
\end{pmatrix}
*
\begin{pmatrix}
2 & -1 & 5 \\
3 & 2 & 4
\end{pmatrix}
=
\begin{pmatrix}
11 & 5 & 17 \\
22 & 10 & 34
\end{pmatrix}
\end{align*}
\begin{align*}
(A*B)^T =
\begin{pmatrix}
11 & 22 \\
5 & 10 \\
17 & 34
\end{pmatrix}
\end{align*}
\begin{align*}
A^T =
\begin{pmatrix}
1 & 2 \\
3 & 6
\end{pmatrix}
\end{align*}
\begin{align*}
B^T =
\begin{pmatrix}
2 & 3 \\
-1 & 2 \\
5 & 4 
\end{pmatrix}
\end{align*}
\begin{align*}
B^T*A^T =
\begin{pmatrix}
2 & 3 \\
-1 & 2 \\
5 & 4 
\end{pmatrix}
*
\begin{pmatrix}
1 & 2 \\
3 & 6
\end{pmatrix}
=
\begin{pmatrix}
11 & 22 \\
5 & 10 \\
17 & 34
\end{pmatrix}
\end{align*}
\begin{align*}
\Rightarrow (A*B)^T = B^T*A^T
\end{align*} 

\item
\(A*B^T\), \(A^T\) und \(B^T\) berechnet man wie in Aufgabe 2c.\\
\(A^T*B^T\) lässt sich nicht berechnen, da die Anzahl der Spalten von A
nicht mit der Anzahl der Zeilen von B übereinstimmt. \\
Somit macht die Gleichung \((A*B)^T = A^T*B^T\) keinen Sinn.
\end{enumerate}

\item[\textbf{3.}]

\item[\textbf{4.}]
\begin{enumerate}[a)]
\item
$f(f^{-1}(B')) \subseteq B'$\\
Es sei: $b \in f(f^{-1}(B'))$\\
Es gilt: $f^{-1}(B')=\{a\}$ mit $a\in A$ für das gilt: es gibt ein $b
\in B'$ mit $f(a)=b$.\\
Also gilt $b\in B'$
\end{enumerate}
%Ende aller Aufgaben
\end{enumerate}
\end{document}

