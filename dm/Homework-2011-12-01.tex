\documentclass[a4paper]{scrartcl}
\usepackage[ngerman]{babel}
\usepackage[utf8]{inputenc}
\usepackage[T1]{fontenc}
\usepackage{lmodern}
\usepackage{amssymb}
\usepackage{amsmath}
\usepackage{enumerate}
\usepackage{scrpage2}\pagestyle{scrheadings}
\usepackage{tikz}

\newcommand{\titleinfo}{Hausaufgaben zum 1./2. Dezember 2011}

\title{\titleinfo}
\author{Elena Noll, Sven-Hendrik Haase, Arne Feil}
\date{\today}
\ihead{EN, SHH, AF}
\chead{\titleinfo}
\ohead{\today}
\setheadsepline{1pt}
\newcommand{\qed}{\quad \square}

\begin{document}
\maketitle

\begin{enumerate}
\item[\textbf{1.}]
\begin{enumerate}[a)]
\item
$AB:\begin{pmatrix}
7 & 5 & -2 \\
2 & 1 & -1 \\
30 & 17 & 5 \\
3 & 2 & 3
\end{pmatrix}$

BA: geht nicht

AC: geht nicht

$AD:\begin{pmatrix}
2 \\
4 \\
26 \\
-4
\end{pmatrix}$

AA: geht nicht

$BB:\begin{pmatrix}
10 & 5 & 1 \\
5 & 4 & -1 \\
4 & 2 & 1
\end{pmatrix}$

$CD:\begin{pmatrix}
12
\end{pmatrix}$

$DC:\begin{pmatrix}
2 & 4 & -4 \\
3 & 6 & -6 \\
-2 & -4 & 4
\end{pmatrix}$

\item
Element, das in AB an dritter Zeile und zweiter Spalte steht: 15

Vierte Spalte von AB: $\begin{pmatrix}
13 \\
8 \\
3 \\
23
\end{pmatrix}$
\end{enumerate}

\item[\textbf{2.}]
\begin{enumerate}[a)]
\item
\begin{align}
A(B_1+B_2)
& = A\begin{pmatrix}
2 & 0 \\
6 & 8
\end{pmatrix}\\ 
& = \begin{pmatrix}
52 & 56 \\
12 & -8 \\
28 & 16
\end{pmatrix}
\end{align}

\[
AB_1 = \begin{pmatrix}
26 & 52 \\
6 & 12 \\
14 & 28
\end{pmatrix}
\]

\[
AB_2 = \begin{pmatrix}
26 & 4 \\
6 & -20 \\
14 & -12
\end{pmatrix}
\]

\[AB_1+AB_2=A(B_1+B_2)\]

\end{enumerate}
\item[\textbf{3.}]

\item[\textbf{4.}]
\begin{enumerate}[a)]
\item
$f(f^{-1}(B')) \subseteq B'$\\
Es sei: $b \in f(f^{-1}(B'))$\\
Es gilt: $f^{-1}(B')=\{a\}$ mit $a\in A$ für das gilt: es gibt ein $b
\in B'$ mit $f(a)=b$.\\
Also gilt $b\in B'$
\end{enumerate}
%Ende aller Aufgaben
\end{enumerate}
\end{document}
