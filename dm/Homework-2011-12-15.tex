\documentclass[a4paper]{scrartcl}
\usepackage[ngerman]{babel}
\usepackage[utf8]{inputenc}
\usepackage[T1]{fontenc}
\usepackage{lmodern}
\usepackage{amssymb}
\usepackage{amsmath}
\usepackage{enumerate}
\usepackage{scrpage2}\pagestyle{scrheadings}
\usepackage{tikz}

\newcommand{\titleinfo}{Hausaufgaben zum 15./16. Dezember 2011}

\title{\titleinfo}
\author{Elena Noll, Sven-Hendrik Haase, Arne Feil}
\date{\today}
\ihead{EN, SHH, AF}
\chead{\titleinfo}
\ohead{\today}
\setheadsepline{1pt}
\newcommand{\qed}{\quad \square}

\begin{document}
\maketitle

\begin{enumerate}
\item[\textbf{1.}]
\begin{enumerate}[a)]
\item Nicht isomorph, denn G hat keinen Knoten 3. Grades, der einen benachbarten Knoten 3. Grades hat. Bei G' ist dies aber der Fall.
\item Alle isomorph, da es sich um Varianten des Petersen-Graphen handelt.
\end{enumerate}
\item[\textbf{2.}]
\begin{enumerate}[a)]
\item \({10 \choose 2}\) = 45
\item \(\sum\limits_{i=1}^{i-3+1}i!\)
\item \(\sum\limits_{i=1}^{i-4+1}i!\)
\item 
\end{enumerate}
\item[\textbf{3.}]
\begin{enumerate}[a)]
\item
\end{enumerate}
\item[\textbf{4.}]
\begin{enumerate}[a)]
\item
\end{enumerate}
%Ende aller Aufgaben
\end{enumerate}
\end{document}

