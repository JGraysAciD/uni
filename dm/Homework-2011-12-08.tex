\documentclass[a4paper]{scrartcl}
\usepackage[ngerman]{babel}
\usepackage[utf8]{inputenc}
\usepackage[T1]{fontenc}
\usepackage{lmodern}
\usepackage{amssymb}
\usepackage{amsmath}
\usepackage{enumerate}
\usepackage{scrpage2}\pagestyle{scrheadings}
\usepackage{tikz}

\newcommand{\titleinfo}{Hausaufgaben zum 8./9. Dezember 2011}

\title{\titleinfo}
\author{Elena Noll, Sven-Hendrik Haase, Arne Feil}
\date{\today}
\ihead{EN, SHH, AF}
\chead{\titleinfo}
\ohead{\today}
\setheadsepline{1pt}
\newcommand{\qed}{\quad \square}

\begin{document}
\maketitle

\begin{enumerate}
\item[\textbf{1.}]
Alle Rechnungen sind in $\mathbb{Z}_{2413}$
\begin{enumerate}[a)]
\item
Bestimmen des $ggT$ mit dem Euklidischen Algorithmus:\\
$2413 = 5 \cdot 473 + 48\\
  473 = 9 \cdot 48 + 1\\
   48 = 1 \cdot 41 + 7\\
   41 = 5 \cdot 7 + 6 \\
    7 = 1 \cdot 6 + 1$ \\
Berechnen des Inversen mit dem erweiterten Euklidischen Algorithmus:\\
\begin{tabular}{rrrr}
2413 & & 1 & 0 \\
473 & & 0 & 1 \\
48 & 5 & 1 & -5 \\
41 & 9 & -9 & 46 \\
7 & 1 & 10 & -51 \\
6 & 5 & -59 & 301 \\
1 & 1 & 69 & \underline{\underline{-352}}
\end{tabular} \\
Da $x \in \{1,2,...,2412\}$ sein soll ist $x = -352 = 2061$.
\item
$2413 = 1 \cdot 1672 + 741 \\
 1672 = 2 \cdot 741 + 190 \\
  741 = 3 \cdot 190 + 171 \\
  190 = 1 \cdot 171 + 19 \\
  171 = 9 \cdot 19 + 0$\\
Da $ggT(2413,1672)=19$ ist, hat 1672 kein Inverses.
\item
Das Inverse von 2412 ist 2412, da $2412 = -1$ und $-1 \cdot-1 = 1$.
\end{enumerate}
\item[\textbf{2.}]
$3^{1000} = (3^{18})^{55} \cdot 3^{10} \quad \quad|\quad 3^{18} \equiv 1~(mod~19)$ Satz von Fermat.\\
$1 \cdot 3^10 = 1 \cdot 3^4 \cdot 3^4 \cdot 3^2\\
\equiv 1 \cdot 5 \cdot 5 \cdot 9 ~(mod~19) = 1 \cdot 25 \cdot 9\\
\equiv 1 \cdot 6 \cdot 9 ~(mod~19)= 54\\
\equiv 16~(mod~19)\\
3^{1000}\equiv 16~(mod~19)$
\newpage
\item[\textbf{3.}]
\begin{enumerate}[a)]
\item

\item

\item
\end{enumerate}
\item[\textbf{4.}]
\begin{enumerate}[a)]
\item

\item

\end{enumerate}
%Ende aller Aufgaben
\end{enumerate}
\end{document}

