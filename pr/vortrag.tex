\documentclass{beamer}
\pdfmapfile{+sansmathaccent.map}
\usetheme{Warsaw}
\begin{document}

\title{Erkennung von Toren beim RoboCup}
\date{2012-07-03}
\author{Elena Noll \and Sven-Hendrik Haase}

\begin{frame}
    \titlepage
\end{frame}

\section*{Gliederung}
\begin{frame}{Gliederung}
    \tableofcontents
\end{frame}

\section{Tore im RoboCup}
\subsection{Geschichte}
\begin{frame}{Tore im RoboCup}{Geschichte}
\begin{itemize}
    \item STL: seit 2012 erstmals gleichfarbige Tore (gelb)
    \item HL: unterschiedliche Farben (gelb/blau)
    \item MSL: weisse Tore
\end{itemize}
\end{frame}

\subsection{Probleme}
\begin{frame}{Tore im RoboCup}{Probleme}
\begin{itemize}
    \item Deformation je nach Perspektive
    \item Probleme bei Lokalisation durch gleichfarbige Tore
    \item Netze können als Feldlinien erkannt werden
    \item Teile vom Tor können verdeckt sein
    \item Torwart im Tor
\end{itemize}
\end{frame}

\section{Bilderkennung}
\subsection{Prozesse}
\begin{frame}{Bilderkennung}{Prozesse}
\begin{enumerate}
    \item Farbkalibrierung und -segmentierung
    \item Erkennung des Horizonts
    \item Extraktion Torpfostenpixel
    \item Modellierung des Tors durch einfach geometrische Formen
\end{enumerate}
\end{frame}

\section*{Quellenverzeichnis}
\begin{frame}{Quellenverzeichnis}
\begin{tiny}
\begin{itemize}
    \item Official RoboCup Standard Platform League Rules for 2010/2011/2012  (http://wiki.robocup.org/wiki/Standard\_Platform\_League\#Rules)
    \item Official RoboCup Humanoid League Rules for 2009/2010/2011/2012 (http://wiki.robocup.org/wiki/Humanoid\_League\#Rules)
    \item Official RoboCup Middle Size League Rules for 2009/2010/2011/2012 (http://wiki.robocup.org/wiki/Middle\_Size\_League\#Rules)
    \item Recognition of Standard Platform RoboCup Goals (http://gsyc.es/jmplaza/papers/jopha-2010.pdf)
\end{itemize}
\end{tiny}
\end{frame}

\end{document}